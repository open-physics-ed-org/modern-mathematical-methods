\section{27 Nov 23 - Notes: Monte Carlo
Simulations}\label{nov-23---notes-monte-carlo-simulations}

\textbf{Monte Carlo Simulations} are a class of computational algorithms
that rely on repeated random sampling to obtain numerical results. The
method is named after the Monte Carlo Casino in Monaco, reflecting its
foundational principle of randomness and chance. Initially developed as
part of the Manhattan Project during World War II, Monte Carlo
simulations have since evolved into a crucial tool in both physics and a
myriad of other fields.

\subsection{Overview of Monte Carlo
Simulations}\label{overview-of-monte-carlo-simulations}

At its core, a
\href{https://en.wikipedia.org/wiki/Monte_Carlo_method}{Monte Carlo
simulation} involves using randomness to solve problems that might be
deterministic in principle. It's particularly useful for systems with a
large number of variables, where traditional analytical methods become
impractical.

In physics, Monte Carlo methods are indispensable for studying complex
systems where analytical solutions are unattainable. They are widely
used in statistical physics, quantum mechanics, and astrophysics, among
other areas. For instance, as we will show they can help model the
behavior of gases, liquids, and solids at the atomic level, and simulate
the evolution of stars and galaxies. In mathematics, these methods offer
a way to approximate solutions to complex integrals and differential
equations. While our primary focus here is on physics and astronomy,
it's worth noting that Monte Carlo methods have been successfully
applied in numerous other domains. For instance, in finance, they are
used to model and predict stock market behaviors and in risk analysis.
In engineering, they help in reliability analysis and optimization
problems. Even in healthcare, they assist in radiation therapy planning
and epidemiological studies.

\subsection{How Monte Carlo Simulations
Work}\label{how-monte-carlo-simulations-work}

\emph{We will go into this later in detail}

Monte Carlo simulations are grounded in the principle of random sampling
and uses the power of statistical analysis to solve problems that, to be
honest, the have no business working for. MC sims use randomness to
explore a large set of possible solutions to a problem, where direct
computation would be impractical or impossible. They then find the most
likely outcomes and use them to approximate the true solution.

A key concept for this work is the use of
\href{https://en.wikipedia.org/wiki/Probability_distribution}{probability
distributions}. In many physical problems, certain outcomes are more
likely than others, and these probabilities can be used to guide the
random sampling process. As westarted to see with microstates and
macrostates -- at thermal equilibrium the most probable macrostate is
easily determined.

\subsubsection{Boltzmann Distribution}\label{boltzmann-distribution}

For example, in thermal physics, the distribution of particle energies
can often be described by the
\href{https://en.wikipedia.org/wiki/Boltzmann_distribution}{Boltzmann
distribution}. This distribution tells us that the probability of a
particle having a certain energy is proportional to the exponential of
the negative of that energy divided by the temperature. This
distribution is used in Monte Carlo simulations to determine the
probability of a particle having a certain energy. The Boltzmann
distribution is given by:

\[P(E) = \frac{e^{-E/kT}}{Z},\]

where \(E\) is the energy state, \(k\) is the Boltzmann constant, \(T\)
is the temperature, and \(Z\) is the
\href{https://en.wikipedia.org/wiki/Partition_function_(statistical_mechanics)}{partition
function}.

\subsubsection{Lecture Video}\label{lecture-video}

This introduction aims to provide a foundational understanding of Monte
Carlo methods; we will delve deeper into the workings, applications, and
mathematical aspects of these simulations.

\href{https://inv.tux.pizza/watch?v=7ESK5SaP-bc}{\pandocbounded{\includegraphics[keepaspectratio,alt={Monte Carlo Simulations}]{https://markdown-videos-api.jorgenkh.no/youtube/7ESK5SaP-bc?width=720&height=405}}}

\begin{itemize}
\tightlist
\item
  Non-Commercial Link: \url{https://inv.tux.pizza/watch?v=7ESK5SaP-bc}
\item
  Commercial Link: \url{https://youtube.com/watch?v=7ESK5SaP-bc}
\end{itemize}

\subsubsection{Basic Components of a Monte Carlo
Simulation}\label{basic-components-of-a-monte-carlo-simulation}

The basic elements of any Monte Carlo Simulation are as follows:

\begin{enumerate}
\def\labelenumi{\arabic{enumi}.}
\tightlist
\item
  \textbf{Random Number Generators (RNGs)}: These are algorithms that
  generate sequences of numbers that approximate the properties of
  random numbers. RNGs are the backbone of any Monte Carlo simulation.
  They are used to generate random samples from probability
  distributions, which are then used to approximate the solution to a
  problem.
\item
  \textbf{Simulation Algorithms}: An example is the
  \href{https://en.wikipedia.org/wiki/Metropolis-Hastings_algorithm}{\textbf{Metropolis-Hastings
  algorithm}}, widely used in statistical physics. This algorithm
  decides whether to accept or reject a new state based on a probability
  criterion, allowing for a detailed exploration of the state space of a
  system.
\end{enumerate}

In the Metropolis-Hastings algorithm, the probability of moving from a
state \(i\) to a state \(j\) is given by:

\[P(i \rightarrow j) = \min\left(1, \frac{e^{-E_j/kT}}{e^{-E_i/kT}}\right),\]

where \(E_i\) and \(E_j\) are the energies of states \(i\) and \(j\)
respectively.

\subsection{Applications in Physics and
Astronomy}\label{applications-in-physics-and-astronomy}

Monte Carlo simulations are widely used in physics and astronomy, where
they help in modeling complex systems and phenomena. Here are some
examples:

\subsubsection{Statistical Physics}\label{statistical-physics}

\begin{itemize}
\tightlist
\item
  \textbf{Phase Transitions and Critical Phenomena}: Monte Carlo
  simulations are instrumental in studying phase transitions, like the
  transition from a ferromagnetic to a paramagnetic state, by allowing
  for the exploration of large lattice systems.
\item
  \textbf{Ising Model of Ferromagnetism}: This model uses a lattice
  where each site has a spin that interacts with its neighbors. Monte
  Carlo methods help in simulating and understanding the magnetic
  properties of materials.
\end{itemize}

\subsubsection{Quantum Mechanics}\label{quantum-mechanics}

\begin{itemize}
\tightlist
\item
  \textbf{Quantum Monte Carlo Methods}: These are used to study systems
  of many interacting quantum particles, providing insights into the
  ground state and excited state properties.
\item
  \textbf{Applications in Atomic and Molecular Physics}: Monte Carlo
  methods help in calculating the properties of atoms and molecules,
  which are otherwise difficult due to the complexity of quantum
  interactions.
\end{itemize}

\subsubsection{Astronomy and
Astrophysics}\label{astronomy-and-astrophysics}

\begin{itemize}
\tightlist
\item
  \textbf{Stellar Evolution Simulations}: By simulating the life cycles
  of stars, Monte Carlo methods help in understanding phenomena like
  supernovae, neutron stars, and black hole formation.
\item
  \textbf{Galactic Dynamics and Dark Matter Modeling}: They are used to
  simulate galaxy formation and the distribution of dark matter in the
  universe.
\end{itemize}

\subsubsection{Cross-disciplinary
Examples}\label{cross-disciplinary-examples}

\begin{itemize}
\tightlist
\item
  \textbf{Climate Modeling}: Monte Carlo simulations are used to predict
  climate change by accounting for the numerous variables and
  uncertainties in climate systems.
\item
  \textbf{Biological Systems}: In biology, they assist in modeling
  complex systems like protein folding and the spread of diseases.
\end{itemize}

\subsection{A Common Example: Estimating
Pi}\label{a-common-example-estimating-pi}

The two most common examples of Monte Carlo are:

\begin{itemize}
\tightlist
\item
  \textbf{Estimating Pi}: A simple application of Monte Carlo is the
  estimation of \(\pi\) by randomly placing points in a square and
  counting how many fall inside a quarter circle.
\item
  \textbf{Roulette Simulations}: A basic gambling game like roulette can
  be simulated to understand probabilities and expected outcomes in
  games of chance.
\end{itemize}

To illustrate the basic principles of Monte Carlo simulations, let's
look at estimating the value of \(\pi\). The code below drops dots on a
plane and determines if the dots are inside or outside a circle. The
ratio of dots inside the circle to the total number of dots is then used
to approximate the value of \(\pi\). The sliders on the widget
demonstrate how the accuracy of the approximation improves with the
number of dots.

\begin{Shaded}
\begin{Highlighting}[]
\ImportTok{import}\NormalTok{ matplotlib.pyplot }\ImportTok{as}\NormalTok{ plt}
\ImportTok{import}\NormalTok{ numpy }\ImportTok{as}\NormalTok{ np}
\ImportTok{import}\NormalTok{ ipywidgets }\ImportTok{as}\NormalTok{ widgets}
\ImportTok{from}\NormalTok{ IPython.display }\ImportTok{import}\NormalTok{ display}
\end{Highlighting}
\end{Shaded}

\begin{Shaded}
\begin{Highlighting}[]
\KeywordTok{def}\NormalTok{ monte\_carlo\_pi\_plot(num\_samples):}
\NormalTok{    inside\_circle }\OperatorTok{=}\NormalTok{ []}
\NormalTok{    outside\_circle }\OperatorTok{=}\NormalTok{ []}

    \ControlFlowTok{for}\NormalTok{ \_ }\KeywordTok{in} \BuiltInTok{range}\NormalTok{(num\_samples):}
\NormalTok{        x, y }\OperatorTok{=}\NormalTok{ np.random.uniform(}\OperatorTok{{-}}\DecValTok{1}\NormalTok{, }\DecValTok{1}\NormalTok{), np.random.uniform(}\OperatorTok{{-}}\DecValTok{1}\NormalTok{, }\DecValTok{1}\NormalTok{)}
        \ControlFlowTok{if}\NormalTok{ x}\OperatorTok{**}\DecValTok{2} \OperatorTok{+}\NormalTok{ y}\OperatorTok{**}\DecValTok{2} \OperatorTok{\textless{}=} \DecValTok{1}\NormalTok{:}
\NormalTok{            inside\_circle.append((x, y))}
        \ControlFlowTok{else}\NormalTok{:}
\NormalTok{            outside\_circle.append((x, y))}

\NormalTok{    estimated\_pi }\OperatorTok{=} \DecValTok{4} \OperatorTok{*} \BuiltInTok{len}\NormalTok{(inside\_circle) }\OperatorTok{/}\NormalTok{ num\_samples}

    \CommentTok{\# Plotting}
\NormalTok{    fig, ax }\OperatorTok{=}\NormalTok{ plt.subplots()}
\NormalTok{    circle }\OperatorTok{=}\NormalTok{ plt.Circle((}\DecValTok{0}\NormalTok{, }\DecValTok{0}\NormalTok{), }\DecValTok{1}\NormalTok{, color}\OperatorTok{=}\StringTok{\textquotesingle{}black\textquotesingle{}}\NormalTok{, fill}\OperatorTok{=}\VariableTok{False}\NormalTok{)}
\NormalTok{    ax.add\_artist(circle)}
\NormalTok{    ax.set\_xlim(}\OperatorTok{{-}}\DecValTok{1}\NormalTok{, }\DecValTok{1}\NormalTok{)}
\NormalTok{    ax.set\_ylim(}\OperatorTok{{-}}\DecValTok{1}\NormalTok{, }\DecValTok{1}\NormalTok{)}
\NormalTok{    ax.set\_aspect(}\StringTok{\textquotesingle{}equal\textquotesingle{}}\NormalTok{, adjustable}\OperatorTok{=}\StringTok{\textquotesingle{}box\textquotesingle{}}\NormalTok{)}
\NormalTok{    ax.scatter(}\OperatorTok{*}\BuiltInTok{zip}\NormalTok{(}\OperatorTok{*}\NormalTok{inside\_circle), color}\OperatorTok{=}\StringTok{\textquotesingle{}blue\textquotesingle{}}\NormalTok{, s}\OperatorTok{=}\DecValTok{1}\NormalTok{)}
\NormalTok{    ax.scatter(}\OperatorTok{*}\BuiltInTok{zip}\NormalTok{(}\OperatorTok{*}\NormalTok{outside\_circle), color}\OperatorTok{=}\StringTok{\textquotesingle{}red\textquotesingle{}}\NormalTok{, s}\OperatorTok{=}\DecValTok{1}\NormalTok{)}
\NormalTok{    ax.set\_title(}\SpecialStringTok{f"Monte Carlo Estimation of Pi: }\SpecialCharTok{\{}\NormalTok{estimated\_pi}\SpecialCharTok{:.4f\}}\SpecialStringTok{ (Samples: }\SpecialCharTok{\{}\NormalTok{num\_samples}\SpecialCharTok{\}}\SpecialStringTok{)"}\NormalTok{)}
\NormalTok{    plt.show()}
\end{Highlighting}
\end{Shaded}

\begin{Shaded}
\begin{Highlighting}[]
\CommentTok{\# Create a slider to control the number of samples}
\NormalTok{slider }\OperatorTok{=}\NormalTok{ widgets.IntSlider(}
\NormalTok{    value}\OperatorTok{=}\DecValTok{100}\NormalTok{,}
    \BuiltInTok{min}\OperatorTok{=}\DecValTok{100}\NormalTok{,}
    \BuiltInTok{max}\OperatorTok{=}\DecValTok{10000}\NormalTok{,}
\NormalTok{    step}\OperatorTok{=}\DecValTok{100}\NormalTok{,}
\NormalTok{    description}\OperatorTok{=}\StringTok{\textquotesingle{}Samples:\textquotesingle{}}\NormalTok{,}
\NormalTok{    continuous\_update}\OperatorTok{=}\VariableTok{False}
\NormalTok{)}

\CommentTok{\# Create an interactive widget}
\NormalTok{widgets.interactive(monte\_carlo\_pi\_plot, num\_samples}\OperatorTok{=}\NormalTok{slider)}
\end{Highlighting}
\end{Shaded}

\begin{verbatim}
interactive(children=(IntSlider(value=100, continuous_update=False, description='Samples:', max=10000, min=100…
\end{verbatim}

\subsection{Additional Resoures}\label{additional-resoures}

\subsubsection{Lecture Video}\label{lecture-video-1}

If you want to see more about how Monte Carlo simulations can work, in
particular, with respect to data science, check out this video.

\href{https://inv.tux.pizza/watch?v=EaR3C4e600k}{\pandocbounded{\includegraphics[keepaspectratio,alt={Monte Carlo Simulations in Data Science}]{https://markdown-videos-api.jorgenkh.no/youtube/EaR3C4e600k?width=720&height=405}}}

\begin{itemize}
\tightlist
\item
  Non-Commercial Link: \url{https://inv.tux.pizza/watch?v=EaR3C4e600k}
\item
  Commercial Link: \url{https://youtube.com/watch?v=EaR3C4e600k}
\end{itemize}
