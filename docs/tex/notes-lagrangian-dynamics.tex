Book Reading: - Any Classical Mechanics Book Section on Lagrangian
Dynamics -
\href{https://www.physics.rutgers.edu/~shapiro/507/book.pdf}{{[}Section
2.1 and 2.2 of Shapiro's Book is online{]}}

\section{5 Sept 23 - Notes: Lagrangian
Dynamics}\label{sept-23---notes-lagrangian-dynamics}

\href{https://en.wikipedia.org/wiki/Newton\%27s_laws_of_motion}{Newtonian
Mechanics} is an incredibly useful model of the natural world. In fact,
it wasn't until the mid 1970s that we were able to truly
\href{https://en.wikipedia.org/wiki/Tests_of_general_relativity}{test
Einstein's gravity as a true replacement} for Newton. That being said,
for most terrestrial situations (macroscopic objects moving at low
speeds), Newton's mechanics is very good. However, the problem with
Newton is that it requires a few things:

\begin{enumerate}
\def\labelenumi{\arabic{enumi}.}
\tightlist
\item
  We must be able identify each interaction on the object or model an
  average behavior from many littler interactions (e.g., models of
  friction vs.~detailed E\&M forces)
\item
  We must be able to mathematically describe the size and direction of
  the interaction at all times we want to model
\item
  We must be able to vectorially add the interactions to produce the net
  force \(\sum_i \vec{F}_i = \vec{F}_{net}\).
\end{enumerate}

In many cases, we can do this. But consider the picture below of a bead
sliding inside a cone. How would you write down the contact force
between the cone and the bead for all space and time?

\begin{figure}
\centering
\pandocbounded{\includegraphics[keepaspectratio,alt={Cone}]{../images/notes-lagrangian-dynamics_cylindrical_cone_mass.jpg}}
\caption{Cone}
\end{figure}

Enter the
\href{https://en.wikipedia.org/wiki/Lagrangian_mechanics}{Lagrangian
Mechanics}, an equivalent description of Newton's mechanics developed
through several advances in physics and mathematics in the 17th and 18th
centuries. Some of the major contributors included Newton, Gottfried
Wilhelm Leibniz, Pierre Louis Moreau de Maupertuis, Guillaume de
l'Hôpital, Jacques Bernoulli, Jean Bernoulli, and Jean D'Alembert. This
period of the development of mechanics was foundational to the
development of concepts like the
\href{https://en.wikipedia.org/wiki/Action_(physics)}{action} and
\href{https://en.wikipedia.org/wiki/Phase_space}{phase space} as well as
approaches like
\href{https://en.wikipedia.org/wiki/Calculus_of_variations}{variational
analysis}, which formed the basis for
\href{https://en.wikipedia.org/wiki/Dynamical_system}{dynamical systems}
(phase space describes real motion),
\href{https://en.wikipedia.org/wiki/Quantum_mechanics}{quantum
mechanics} (quantize the action), and
\href{https://en.wikipedia.org/wiki/Perturbation_theory}{perturbation
theory} (approximate answers can be iteratively sought).

This analysis is grounded in optimization or rather extremization. The
idea is as follows:

\begin{enumerate}
\def\labelenumi{\arabic{enumi}.}
\tightlist
\item
  Consider all potential paths a system can take through phase space
\item
  Time each of them between the same two points in phase space
\item
  The one that minimizes the action integral over that time/space is the
  one the system takes
\end{enumerate}

This might seem magical! But it's truly a deep connection to the
energetics of the system, which limit the the relevant equations of
motion by both the number of degrees of freedom (ways the system can
move) and the constraints equation (things that influence the motion).

\subsection{Variational Analysis}\label{variational-analysis}

The name of the game in calculus of variations is finding extrema
(minima, maxima, or stationary points) of integrals that have the form:

\[
S = \int_{x_1}^{x_2} f[y(x),\dot{y}(x),x] dx
\]

While at first this might seem like a strange thing to do, it turns out
that this is a very powerful way to solve problems. In fact, we are able
to characterize the path of a system through phase space by finding the
path that minimizes the
\href{https://en.wikipedia.org/wiki/Action_(physics)}{action integral} .
This is called the
\href{https://en.wikipedia.org/wiki/Stationary-action_principle}{principle
of least action}. This kind of analysis also tells us that a
\href{https://en.wikipedia.org/wiki/Great_circle}{Great Circle} is the
shortest path between two points on a sphere and that
\href{https://en.wikipedia.org/wiki/Snell\%27s_law}{Snell's Law} is the
shortest path (in time) light takes between two points in different
media.

While you are trying to find the minimize \(S\), what you end up finding
is the \textbf{function} \(y(x)\) that satisfies this minimization. It
turns out that for \(S\) to have extrema, the Euler-Lagrange equation
(below) must be satisfied. The handwritten notes below show how to
derive this equation for a 1D system, but it can be generalized to \(N\)
dimensions.

\[
\frac{\partial f}{\partial y} - \frac{d}{dx}\left(\frac{\partial f}{\partial \dot{y}} \right) = 0
\]

\subsubsection{Applications to Classical
Mechanics}\label{applications-to-classical-mechanics}

In practice, when approaching a variational problem, the typical
workflow if something like this:

\begin{enumerate}
\def\labelenumi{\arabic{enumi}.}
\tightlist
\item
  Write your problem down in the form of an integral like \(S\).
\item
  Use the Euler-Lagrange equation to get a differential equation for the
  unknown function \(y\).
\item
  Solve the differential equation.
\end{enumerate}

We can extend this framework for use in classical mechanics by defining
the lagrangian of a system with independent, generalized coordinates
\((q_1,\dot{q}_1... q_n,\dot{q}_n)\) as the kinetic energy minus
potential energy of a system:

\[
\mathcal{L(q_1,\dot{q}_1... q_n,\dot{q}_n)} = T(q_1,\dot{q}_1... q_n,\dot{q}_n) - V(q_1,\dot{q}_1... q_n,\dot{q}_n)
\]

Here, the function \(\mathcal{L}\) is called the Lagrangian of the
system and takes the place of \(f\) in the Euler-Lagrange equation. The
coordinates \(q_i\) are called the generalized coordinates of the system
and they take the place of \(y\) in the 1D Euler-Lagrange equation.

The \href{https://en.wikipedia.org/wiki/Action_(physics)}{action} is a
scalar quantity that ``tracks'' how the energy changes in a physical
system over time. It is the integral of the Lagrangian over a time
interval, which we write as the action integral:

\[
S = \int_{t_1}^{t_2} \mathcal{L(q_1,\dot{q}_1... q_n,\dot{q}_n)} dt
\]

Several mathematicians contributed to this work including Leibniz,
BErnoulli, Maupertuis, and Euler. Through their collective work, it
turns out we can think of the path that a system takes through an
abstract space of all the measurements that uniquely characterize it (a
phase space). That trajectory through phase space is just as useful as a
real physical trajectory in \(x\), \(y\), \(z\) space if we understand
what it is doing. And we can abstract that concept to new systems that
might not have mechanical analogs (e.g., quantum mechanics).

Their work showed that the path a system takes between points \(1\) and
\(2\) in these generalized coordinates is the path such that \(S\) is
stationary (you can think of this as minimizing the action integral).
This is called the principle of least action. This lets us leverage the
Euler-Lagrange equation for the generalized coordinates of our system
\(q_n\). We can derive that result, but it is not necessary for you to
do so to see the connection between the equation below and the 1D
Euler-Lagrange equation.

\[
\frac{\partial \mathcal{L}}{\partial q_i} - \frac{d}{dt}\left(\frac{\partial \mathcal{L}}{\partial \dot{q}_i} \right) = 0
\]

Once we perform these derivatives on this function, we will obtain \(n\)
equations of motion (EOM) for our system; one for each generalized
coordinate (\(q_i\)). Note how we didn't have to know anything about the
forces acting on our system to arrive at equations of motion, but rather
the energy.

\subsection{Video on Lagrangian
Dynamics}\label{video-on-lagrangian-dynamics}

Parth G. has a lovely video below about the basics of Lagrangian
Dynamics.

\href{https://inv.tux.pizza/watch?v=KpLno70oYHE}{\pandocbounded{\includegraphics[keepaspectratio]{https://markdown-videos-api.jorgenkh.no/youtube/KpLno70oYHE?width=720&height=405}}}
- Non-Commercial Link: \url{https://inv.tux.pizza/watch?v=KpLno70oYHE} -
Commercial Link: \url{https://youtube.com/watch?v=KpLno70oYHE}

\subsection{Additional Resources}\label{additional-resources}

My notes go into detail on the development of the Lagrangian problem.
But practice is the best approach.

\subsubsection{Handwritten Notes}\label{handwritten-notes}

Here are my handwritten notes on the Calculus of Variations and
Lagrangian Mechanics.

\begin{itemize}
\tightlist
\item
  \href{../../assets/notes/Notes-Calculus_of_Variations.pdf}{Calculus of
  Variations}
\item
  \href{../../assets/notes/Notes-Lagrangian_Dynamics.pdf}{Lagrangian
  Dynamics}
\item
  \href{../../assets/notes/Notes-Lagrangian_Example.pdf}{Lagrangian
  Example}
\item
  \href{../../assets/notes/Notes-Lagrangian_Example_Gen_Force.pdf}{Lagrangian
  Example with Generalized Forces and Lagrange Multipliers}
\end{itemize}

\subsubsection{Book Readings}\label{book-readings}

This reading is useful preparation for reminding yourself of Lagrangian
Dynamics.

\begin{itemize}
\tightlist
\item
  Any classical mechanics book you like! Taylor is a good one.
\item
  \href{https://www.physics.rutgers.edu/~shapiro/507/book.pdf}{Section
  2.1 and 2.2 of Shapiro's Book}
\item
  \href{https://www.google.com/books/edition/A_Student_s_Guide_to_Lagrangians_and_Ham/ebTCAQAAQBAJ?hl=en&gbpv=1&dq=lagrangian+dynamics+book&printsec=frontcover}{A
  Google Book you can review}
\end{itemize}

\subsubsection{Video Resources}\label{video-resources}

If you are a feeling that you would like a little direct instruction on
this, this lecture is great. Lots of examples!

\href{https://inv.tux.pizza/watch?v=zhk9xLjrmi4}{\pandocbounded{\includegraphics[keepaspectratio]{https://markdown-videos-api.jorgenkh.no/youtube/zhk9xLjrmi4?width=720&height=405}}}
- Non-Commercial Link: \url{https://inv.tux.pizza/watch?v=zhk9xLjrmi4} -
Commercial Link: \url{https://youtube.com/watch?v=zhk9xLjrmi4}
