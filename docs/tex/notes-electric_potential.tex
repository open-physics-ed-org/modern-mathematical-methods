\section{3 Oct 23 - Notes: Electric Potential - a scalar
field}\label{oct-23---notes-electric-potential---a-scalar-field}

We've seen that there's a lot of challenges with solving field problems
where the fields are vectors and described by partial differential
equations. The electric and magnetic field are vector fields that have a
size and direction are every points in space. They are described by a
set of 4 partial differential equations,
\href{https://en.wikipedia.org/wiki/Maxwell\%27s_equations}{Maxwell's
Equations}. In Cartesian coordinates, these equations are described by 8
coupled partial differential equations.

\subsection{Electrostatics}\label{electrostatics}

In electrostatic situations, where charges don't move, we've seen that
the equations that describe the electric field are simplified to:

\[\nabla \cdot \vec{E} = \frac{\rho}{\epsilon_0}\]

\[\nabla \times \vec{E} = 0\]

In general, in Cartesian coordinates, these are still four partial
differential equations. However, the electric field is
\href{https://en.wikipedia.org/wiki/Curl_(mathematics)}{curl-less}, so
we can write it as the gradient of a scalar field, \(V\):

\[\vec{E}(x,y,z) = -\nabla V(x,y,z)\]

We might prefer to work with this scalar field, \(V\), rather than the
vector field, \(\vec{E}\), because it is a simpler object to work with.
There's a number of tools we can use to find the potential and then we
take the gradient to find the electric field.

\subsubsection{Electric Potential}\label{electric-potential}

Parth G has a really nice video on the conceptual elements of electric
potential, we will concern ourselves with more mathematical aspects as
we move forward.

\href{https://inv.tux.pizza/watch?v=7rjAtuwxrEA}{\pandocbounded{\includegraphics[keepaspectratio,alt={Electric Potential}]{https://markdown-videos-api.jorgenkh.no/youtube/7rjAtuwxrEA?width=720&height=405}}}

\begin{itemize}
\tightlist
\item
  Non-Commercial Link: \url{https://inv.tux.pizza/watch?v=7rjAtuwxrEA}
\item
  Commercial Link: \url{https://youtube.com/watch?v=7rjAtuwxrEA}
\end{itemize}

The \href{https://en.wikipedia.org/wiki/Electric_potential}{electric
potential} is the scalar field that we have developed. One thing to note
about it, is that it is only defined up to a constant. That is because
when we take the gradient, and constant will drop out. This is really
important when stitching together solutions from multiple sources. They
need to all have the same ``zero of potential.''

With the definition, \(\vec{E}(x,y,z) = -\nabla V(x,y,z)\), we can
conclude the following:

\[\nabla \cdot \vec{E} = \nabla \cdot (-\nabla V) = -\nabla^2 V = \frac{\rho}{\epsilon_0}\]

where \(\nabla^2\) is the
\href{https://en.wikipedia.org/wiki/Laplace_operator}{Laplacian
operator}. This is
\href{https://en.wikipedia.org/wiki/Poisson\%27s_equation}{Poisson's
Equation}, a well-studied second order partial differential equation.
When used in electrostatics, it is a central equation for solving for
the electric potential. In addition, we can integrate to show that:

\[\Delta V = -\int \vec{E} \cdot d\vec{l}\]

where the line integral is taken along a path from one point to another.
For the electric field, this is independent of path, so we can write:

\[\Delta V = -\int_{\vec{r}_1}^{\vec{r}_2} \vec{E} \cdot d\vec{l} = V(\vec{r}_2) - V(\vec{r}_1)\]

And

\[\oint \vec{E} \cdot d\vec{l} = 0\]

which is a direct results of applying Stokes' Theorem to the curl-less
electric field, and indicates that the electric field is conservative.

\subsection{Getting to Laplace's
Equation}\label{getting-to-laplaces-equation}

Poisson's Equation is a description of a scalar potential (e.g., the
electric potential) and how it relates to sources of the potential
(e.g., charges). This is a \textbf{local} description of the field. At
every point, the Laplacian can be computed (for well-behaved functions)
and it indicates a source or lack of sources. For example, in empty
space with no charges, the Laplacian is zero everywhere.

So if we imagine setting up charges far from the place we are interested
in finding the electric potential, we have an equation that is describes
that space.

\[\nabla^2 V = 0\]

This is Laplace's equation and is a commonly occurring equation in
physics. In fact, as we saw here, it shows up anytime we find local
situations where the charges are absent. So the idea is that charges are
far away setting the potential at the boundaries (hence, calling these
\textbf{\href{https://en.wikipedia.org/wiki/Boundary_value_problem}{boundary
value problems}}). The idea is that the PDE is solved in the region of
interest often giving some general solution in the form of an infinite
sum. The boundary conditions are then used to determine the coefficients
in the solution, giving a unique solution. Because Laplace's equation is
a linear PDE, the superposition principle applies, so we can add
solutions together to get new solutions.

\subsubsection{A typical problem}\label{a-typical-problem}

It might seem odd to have an equation that is everywhere equal to zero
and somehow suggest it produces non-zero solutions. But remember that it
is a PDE that describes the second derivatives of this scalar function,

\[\nabla^2 V = \frac{\partial^2 V}{\partial x^2} + \frac{\partial^2 V}{\partial y^2} + \frac{\partial^2 V}{\partial z^2} = 0\]

It turns out there's lots of functions that will satisfy that
differential equation. And it's mostly a matter of if the function fits
the boundary conditions. Let's look at a typical problem.

\pandocbounded{\includegraphics[keepaspectratio,alt={Typical Laplace's Equation Problem}]{https://www.physics.uoguelph.ca/sites/default/files/uploads/fig10_1-parallel-plates.jpg}}
\emph{Image from
\href{https://www.physics.uoguelph.ca/chapter-10-laplaces-equation}{University
of Guelph}}

Here is a ``gutter'' where the electric potential for different sides of
the gutter are set. Laplace's equation applies inside the gutter. So the
idea is to find a solution that satisfies Laplace's equation and the
boundary conditions. We will develop this solution in class, but the
representative results look like this:

\pandocbounded{\includegraphics[keepaspectratio,alt={Typical Laplace's Equation Solution}]{https://www.physics.uoguelph.ca/sites/default/files/uploads/fig10_2_potential-plates.jpg}}
\emph{Image from
\href{https://www.physics.uoguelph.ca/chapter-10-laplaces-equation}{University
of Guelph}}

\subsection{Separation of Variables}\label{separation-of-variables}

One of the key analytical tools that we have for Laplace's Equation is
the
\href{https://en.wikipedia.org/wiki/Separation_of_variables}{separation
of variables} method of solving PDEs. We posit an ansatz in which the
solution is a product of functions of the individual variables. For
example, in Cartesian coordinates, we might write:

\[V(x,y,z) = X(x)Y(y)Z(z)\]

And then we plug that into Laplace's equation and see if we can find
solutions. In this case, we get:

\[\nabla^2 V = YZ\frac{d^2 X}{dx^2} + XZ\frac{d^2 Y}{dy^2} + XY\frac{d^2 Z}{dz^2} = 0\]

which we divide by \(XYZ\) to obtain the following expression:

\[\frac{1}{X}\frac{\partial^2 X}{\partial x^2} + \frac{1}{Y}\frac{\partial^2 Y}{\partial y^2} + \frac{1}{Z}\frac{\partial^2 Z}{\partial z^2} = 0\]

Here each partial derivative is a of a pure function of a single
variable. So we can conclude that each term is a constant that sum to
zero. We can write this as:

\[\frac{d^2 X}{d x^2} = -k_x^2X\] \[\frac{d^2 Y}{d y^2} = -k_y^2Y\]
\[\frac{d^2 Z}{d z^2} = -k_z^2Z\]

where \(k_x^2 + k_y^2 + k_z^2 = 0\). We have converted our problem to a
set of 3 ODEs that we know how to solve. We can then solve each of these
equations separately and show a general solution is:

\[X(x) = Ae^{ik_xx} + Be^{-ik_xx}\] \[Y(y) = Ce^{ik_yy} + De^{-ik_yy}\]
\[Z(z) = Fe^{ik_zz} + Ge^{-ik_zz}\]

where \(A,B,C,D,E,F,G\) are constants. We can then use the boundary
conditions to determine the constants. Notice we wrote these as complex
exponentials, those are sinusoidal or exponential solutions depending on
if the resulting \(k\)'s are real or complex. The electric potential is
a real measurable quantity.

This general approach is well known and well-traveled territory. It's
working with these boundary conditions that can be tricky, and that we
will practice. We will find that for most cases that we end up with
infinite sums of these solutions. We will also find that we can use the
\href{https://en.wikipedia.org/wiki/Orthogonal_functions}{orthogonality
of the solutions} to simplify the solutions.

\textbf{We should not suggest that every PDE can be solved with
separation of variables, but rather that:} * Separation of variables is
a powerful tool for solving PDEs that you can try, but it might not
work. * It is a tool that can produce general solutions to Laplace's
Equation when the Laplacian is separable in the given coordinate system.
* It cannot be used for some problems (e.g., when the boundary
conditions cannot be expressed as
\href{https://en.wikipedia.org/wiki/Neumann_boundary_condition}{Neumann}
or
\href{https://en.wikipedia.org/wiki/Dirichlet_boundary_condition}{Dirichlet}
boundary conditions).

\subsubsection{Other Coordinate Systems}\label{other-coordinate-systems}

It turns out the trick of positing a solution that is a product of
functions of the individual variables works in other coordinate systems
as well. For example, in cylindrical coordinates, we might write:

\[V(r,\theta,z) = R(r)\Theta(\theta)Z(z)\]

Or in spherical coordinates, we might write:

\[V(r,\theta,\phi) = R(r)\Theta(\theta)\Phi(\phi)\]

It turns out both of these are separable in their respective coordinate
systems. We will work through some examples of spherical coordinates in
class as it forms the basis of the
\href{https://en.wikipedia.org/wiki/Multipole_expansion}{Multipole
Expansion}, which is a critical mathematical tool with wide
applications.

This idea of funding general solutions to Laplace's equation in known
orthogonal coordinate systems remains incredible common. Experimental
setups and theoretical models often have situations where Laplace's
equation is relevant and there is a symmetry that can be exploited to
find solutions. Laplace's equation is separable in 13 coordinate
systems. You can find some of the more
\href{https://en.wikipedia.org/wiki/Orthogonal_coordinates\#Table_of_three-dimensional_orthogonal_coordinates}{exotic
equations and coordinate systems here}. Check out the
\href{https://en.wikipedia.org/wiki/Prolate_spheroidal_coordinates}{Prolate
Spheroidal coordinates}, which are used in nuclear physics when the
nucleus is deformed.

\begin{figure}
\centering
\pandocbounded{\includegraphics[keepaspectratio,alt={Prolate Spheroidal Coordinates}]{https://upload.wikimedia.org/wikipedia/commons/b/be/Prolate_spheroidal_coordinates.png}}
\caption{Prolate Spheroidal Coordinates}
\end{figure}

\subsection{Additional Resources}\label{additional-resources}

\subsubsection{Textbook Chapter}\label{textbook-chapter}

The University of Guelph provides free notes in physics. Here's the
\href{https://www.physics.uoguelph.ca/chapter-10-laplaces-equation}{relevant
chapter on Laplace's Equation}. It is also where some of the figures
were taken from.

\subsubsection{Videos Derivations}\label{videos-derivations}

Steve Brunton at Washington has some really great videos on the
derivations associated with PDEs like those above.

\paragraph{Possion's and Laplace's
Equation}\label{possions-and-laplaces-equation}

\href{https://inv.tux.pizza/watch?v=nmvs0vrBT18}{\pandocbounded{\includegraphics[keepaspectratio,alt={Possion's and Laplace's Equation}]{https://markdown-videos-api.jorgenkh.no/youtube/nmvs0vrBT18?width=720&height=405}}}

\begin{itemize}
\tightlist
\item
  Non-Commercial Link: \url{https://inv.tux.pizza/watch?v=nmvs0vrBT18?}
\item
  Commercial Link: \url{https://youtube.com/watch?v=nmvs0vrBT18?}
\end{itemize}

\paragraph{Separation of Variables}\label{separation-of-variables-1}

\href{https://inv.tux.pizza/watch?v=VjWtMl6vQ3Q}{\pandocbounded{\includegraphics[keepaspectratio,alt={Separation of Variables}]{https://markdown-videos-api.jorgenkh.no/youtube/VjWtMl6vQ3Q?width=720&height=405}}}

\begin{itemize}
\tightlist
\item
  Non-Commercial Link: \url{https://inv.tux.pizza/watch?v=VjWtMl6vQ3Q}
\item
  Commercial Link: \url{https://youtube.com/watch?v=VjWtMl6vQ3Q}
\end{itemize}

\subsubsection{Handwritten Notes}\label{handwritten-notes}

\begin{itemize}
\tightlist
\item
  \href{../../assets/notes/Notes-Electric_Potential.pdf}{Electric
  Potential}
\item
  \href{../../assets/notes/Notes-Separation_of_Variables_Cartesian.pdf}{Separation
  of Variables}
\item
  \href{../../assets/notes/Notes-Separation_of_Variables_Spherical.pdf}{Separation
  of Variables (Spherical Coordinates)}
\item
  \href{../../assets/notes/Notes-Multipole_Expansion.pdf}{Multipole
  Expansion}
\end{itemize}
