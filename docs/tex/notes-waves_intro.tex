\section{16 Oct 23 - Notes: Introduction to
Waves}\label{oct-23---notes-introduction-to-waves}

Coupled oscillations are quite common in nature; collections of objects
and larger systems exhibit oscillatory behavior frequently. At times,
the number of objects in a system that are oscillating are so many that
we begin to zoom out and look at the behavior not of individual elements
but of larger collections. This is moving to a continuum view of this
oscillatory behavior. Consider going from 1 SHO, to two coupled SHO, to
N coupled SHOs where N is very large, and, finally to a rope. It seems
like there is a connection between a rope and a set of SHOs. Indeed, we
can develop the wave equation for a rope using this kind of modeling.
That is a well-known result, so we will instead use it. We start with
motivating ourselves with how important waves are to our understanding
of nature.

\href{https://inv.tux.pizza/watch?v=hebGhsNsjG0}{\pandocbounded{\includegraphics[keepaspectratio,alt={Graviational Waves}]{https://markdown-videos-api.jorgenkh.no/youtube/hebGhsNsjG0?width=720&height=405}}}

\begin{itemize}
\tightlist
\item
  Non-Commercial Link: \url{https://inv.tux.pizza/watch?v=hebGhsNsjG0}
\item
  Commercial Link: \url{https://youtube.com/watch?v=hebGhsNsjG0}
\end{itemize}

\subsection{Gravitational Waves}\label{gravitational-waves}

Predicted by the
\href{https://en.wikipedia.org/wiki/General_relativity}{General Theory
of Relativity} in the early 20th centruty, but only
\href{https://news.mit.edu/2016/ligo-first-detection-gravitational-waves-0211}{observed
less than a decade},
\href{https://en.wikipedia.org/wiki/First_observation_of_gravitational_waves}{gravitational
waves} are not only a critical finding in our search to understand the
universe, but also a new {[}way to observe{]} the universe! These waves
ripple through space time, disrupting matter on the smallest of scales.
The LIGO detector's strain sensitivity (how small a deformation it can
detect) is incredible. At scale, LIGO can detect a change in length
about the width of your thumb over a distance that would be between us
and the next closest galaxy. That is incredibly sensitive.

```\{admonition\} Discovery of Gravitation Waves The work to find
measure gravitational waves has been a long effort. Since the prediction
of these waves by Einstein in 1916, folks have tried to push
experimentalists to construct a detector. Enter LIGO, the massive
gravitational wave observatories in Louisiana and Washington State.
These detectors were discussed since the 1960s! But only started to be
developed in the late 1990s. These detectors consist of two 3km long
interferometers that measure the time of flight between ends of the
detectors. The length is needed because the changes are so small! They
came online in 2002, observing the first gravitational wave on 14 Sept
2015 at 5:51AM ET.

One scientist who lead the LIGO team and has been instrumental in the
work of LIGO since then is
\href{https://physics.mit.edu/faculty/nergis-mavalvala/}{Nergis
Mavalvala}. She is a MacArthur award winning scientist who directed the
first successful LIGO mission. While the
\href{https://www.nobelprize.org/prizes/physics/2017/press-release/}{2017
Nobel prize} went to the Rainer Weiss, Barry Barish and Kip Throne for
their pioneering work on gravitational waves including immense
contributions to the design and development of LIGO, it was Mavalvala
who directed the mission. ```

\subsection{The Wave Equation}\label{the-wave-equation}

There are \emph{many} wave equations. They simply describe the
relationship between how something changes in space and time. It is a
specific relationship that relates the influences (e.g., forces,
electric interactions, viscosity, etc.) to what we can observe (e.g.,
changes in height, frequency, color, or number). But the one to being
with is the the most basic one. By basic, we do not mean that it is
simple but rather that it is the basis for many other wave equations.
This is the 1D wave equation for a rope:

\[\dfrac{\partial^2 h(x,t)}{\partial t^2} = \dfrac{T_0}{\rho_0}\dfrac{\partial^2 h(x,t)}{dx^2}\]

The function \(h(x,t)\) represent the height of the string relative to
some zero line both along the strong (in the \(x\) direction) and over
time (as \(t\) marches forward).

Solving this equation in general can be done using a standing wave
solution - the continuous analog to our normal modes approach for
coupled oscillators. We can guess the mode will be oscillatory and
derive the resulting general solution. This is done in the principal
reading below, which comes from Crawford's Waves book.

\href{https://inv.tux.pizza/watch?v=ub7lok-JQJE}{\pandocbounded{\includegraphics[keepaspectratio,alt={Introduction to the Wave}]{https://markdown-videos-api.jorgenkh.no/youtube/ub7lok-JQJE?width=720&height=405}}}

\begin{itemize}
\tightlist
\item
  Non-Commercial Link: \url{https://inv.tux.pizza/watch?v=ub7lok-JQJE}
\item
  Commercial Link: \url{https://youtube.com/watch?v=ub7lok-JQJE}
\end{itemize}

\subsection{Additional Resources}\label{additional-resources}

\subsubsection{Crawford's Waves Book}\label{crawfords-waves-book}

Crawford's book is excellent here, but derives the equation from first
principles. This is important physics, but I do not expect you to
develop such a model from scratch. It is an important derivation that
shows how we start many PDE problems. More important to our work those
is what appears from pages 54 forward. The solutions and their meaning.
We will get to using Fast Fourier Transforms starting from this, but you
don't need to read about the Fourier Analysis yet. Just focus on Secs.
2.1-2.2 and the first parts of 2.3.

\begin{itemize}
\tightlist
\item
  \href{../assets/scans/Crawford-Waves-Sec_2.1-2.3.pdf}{Crawford Sec
  2.1-2.3}
\end{itemize}

\subsubsection{Lecture Video}\label{lecture-video}

The video below covers the Crawford's book in the first 30 minutes. The
instructor speaks quickly, but the content is solid and follows what
you'll read fairly well. If you want more of the mathematical
background, watch this video for sure. We will cover some of it in
class, but you are not expected you to derive it.

\href{https://inv.tux.pizza/watch?v=1JeBWHzrRD4}{\pandocbounded{\includegraphics[keepaspectratio,alt={Waves Lecture}]{https://markdown-videos-api.jorgenkh.no/youtube/1JeBWHzrRD4?width=720&height=405}}}

\begin{itemize}
\tightlist
\item
  Non-Commercial Link: \url{https://inv.tux.pizza/watch?v=1JeBWHzrRD4}
\item
  Commercial Link: \url{https://youtube.com/watch?v=1JeBWHzrRD4}
\end{itemize}
