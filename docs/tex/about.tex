\section{About}\label{about}

This site is maintained by \href{https://dannycab.github.io/}{Danny
Caballero} and is intended to be a resource for teaching and learning
modern classical mechanics. It is dedicated to providing accessible,
interactive, and high-quality materials for students and educators. If
not for the students I've been privileged to teach and learn from over
the years and my colleagues who inspire my teaching this would not have
been started.

\subsection{Pull Requests Welcome!}\label{pull-requests-welcome}

We welcome contributions from anyone including students who want to help
improve the site and its resources. Whether you're a student, educator,
or developer, your input is valuable.

\begin{itemize}
\tightlist
\item
  \href{https://github.com/dannycab/modern-classical-mechanics/issues/new}{Create
  an issue}
\item
  \href{https://github.com/dannycab/modern-classical-mechanics/pulls}{Issue
  a pull request}
\end{itemize}

You don't need a GitHub account to create an issue, but you do need one
to issue a pull request.

\section{Philosophy}\label{philosophy}

\subsection{Why Accessible, Open Educational
Resources?}\label{why-accessible-open-educational-resources}

Our work is motivated by a deep need to challenge the corporate control
of educational resources and to make high-quality, accessible materials
available to everyone.

When corporations control access to textbooks, software, and other
learning tools, only those who can afford them benefit. This creates
unnecessary barriers for students and educators everywhere.

The
\href{https://educationdata.org/average-cost-of-college-textbooks}{rising
costs of textbooks},
\href{https://educationdata.org/college-tuition-inflation-rate}{increasing
tuition}, and
\href{https://www.brookings.edu/articles/increasing-the-impact-of-corporate-engagement-in-education-landscape-and-challenges/}{corporate
influence over educational resources} have made it harder than ever for
people to get the materials they need. By making resources open and
accessible, we hope to lower these barriers and support a more equitable
learning environment for all.

\subsection{Why is the site so plain? Why is it so poorly designed? Why
is it so
simple?}\label{why-is-the-site-so-plain-why-is-it-so-poorly-designed-why-is-it-so-simple}

The challenge of accessible materials is that they must be available to
everyone in a format that is easy to use, modify, and distribute. This
includes considerations for those with disabilities, different needs,
and varying levels of access to technology.

So, we have attempted to create a site and a build that are open,
simple, and accessible. This also means that the format of the materials
is available in many different forms automatically, including HTML, PDF,
and Markdown, DOCX, LaTeX, and Jupyter Notebook formats.

Indeed, even the build process itself is open and accessible, allowing
anyone to generate the site and its resources in the format they prefer.
The entire site is available in a
\href{https://github.com/dannycab/modern-classical-mechanics}{public
GitHub repository}. It's also hosted there.

\emph{A note here:} Danny is not a Python developer; he has a full time
job as a \href{https://dannycab.github.io/}{physics and computational
science professor}). The code is gonna be crap.

\subsection{Design Principles}\label{design-principles}

We abide the following design principles:

\begin{enumerate}
\def\labelenumi{\arabic{enumi}.}
\tightlist
\item
  Everything is open source and free to use, modify, and distribute.
\item
  All materials will follow best guidelines for accessibility, ensuring
  that they are usable by everyone especially those with disabilities.
\item
  Anyone can contribute to the project, whether by suggesting changes,
  adding new content, or improving existing materials. Pull requests are
  welcome!
\end{enumerate}

\section{Accessibility Timeline}\label{accessibility-timeline}

\subsection{Accessibility}\label{accessibility}

We intend these resources to be accessible to everyone and available in
a variety of formats. This work involves ongoing efforts to ensure that
all materials meet accessibility standards and are usable by people with
disabilities. That is a work in progress, and we will continue to
improve the site and its resources over time. Below is the todo list of
our accessibility efforts:

\subsection{Building Accessible
Resources}\label{building-accessible-resources}

\begin{itemize}
\tightlist
\item[$\boxtimes$]
  Build with open source tools that are readily accessible
\item[$\square$]
  Provide documentation and resources for others to contribute
\end{itemize}

\subsection{Web Accessibility
Standards}\label{web-accessibility-standards}

\begin{itemize}
\tightlist
\item[$\square$]
  Ensure all images have appropriate alt text
\item[$\square$]
  Implement ARIA roles and properties correctly
\item[$\square$]
  Use semantic HTML to improve screen reader compatibility
\item[$\square$]
  Ensure all links have descriptive text
\item[$\square$]
  Provide transcripts for all audio and video content
\item[$\square$]
  Build keyboard navigation for all interactive elements
\end{itemize}

\subsection{Accessibility Testing}\label{accessibility-testing}

\begin{itemize}
\tightlist
\item[$\square$]
  Ensure color contrast meets WCAG standards
\item[$\square$]
  Test site with screen readers and other assistive technologies
\item[$\square$]
  Ensure all interactive elements are keyboard accessible
\end{itemize}

\section{Contributions}\label{contributions}

We welcome contributions from anyone including students who want to help
improve the site and its resources. Whether you're a student, educator,
or developer, your input is valuable. You can help by: - Suggesting new
content or improvements to existing materials - Reporting issues or bugs
- Reviewing pull requests from others - Adding new activities or
simulations - Improving the site's design or functionality

You can contribute by creating an issue or issuing a pull request on the
\href{https://github.com/dannycab/modern-classical-mechanics}{GitHub
repository}.

\subsection{Contributors}\label{contributors}

Over the years, the following people have contributed to this project
and its resources:

\begin{itemize}
\tightlist
\item
  Morten Hjorth-Jensen (MSU/UiO)
\item
  Rachel Henderson (MSU)
\item
  Vashti Sawtelle (MSU)
\item
  Steve Pollock (CU-Boulder)
\item
  Alia Valentine (MSU)
\end{itemize}
