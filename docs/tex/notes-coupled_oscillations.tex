\section{16 Oct 23 - Notes: Coupled
Oscillations}\label{oct-23---notes-coupled-oscillations}

Single particle dynamics like the SHO and other singular point particle
motion (e.g., basic kinematics, charges in fields modeled without
interactions) are helpful in building our intuition of how physics
\emph{should} work. We can track the position, velocity, and other
dynamical quantities for these singular particles. We can determine how
the act on and interact with their surroundings. We can find points of
stasis (stable and unstable points); we can find different families of
solutions (orbits vs runaway behavior); we can predict the motion of
those particles with numerical routines.

But what happens when we introduce a second particle? And we let it
interact with the first particle. We have substantially opened the
degrees of freedom (a way to characterize the complexity of our system)
that our system has. We will see how these systems become expanded and
how their dynamics becomes richer and more interesting.

The canonical complication is the coupled oscillators, which will form
the basis of our initial study, much like the SHO formed our initial
work for dynamical models using ODEs. The setup appears below:

We will set up this problem below, but our analysis will proceed in
class. To motivate our study of coupled oscillations, watch this
excellent video (20 minutes) on the phenomenon on
\href{https://physicstoday.scitation.org/doi/10.1063/1.1554136}{synchronization}.

\href{https://inv.tux.pizza/watch?v=t-_VPRCtiUg}{\pandocbounded{\includegraphics[keepaspectratio,alt={Synchronization}]{https://markdown-videos-api.jorgenkh.no/youtube/t-_VPRCtiUg?width=720&height=405}}}

\begin{itemize}
\tightlist
\item
  Non-Commercial Link: \url{https://inv.tux.pizza/watch?v=t-_VPRCtiUg}
\item
  Commercial Link: \url{https://youtube.com/watch?v=t-_VPRCtiUg}
\end{itemize}

\subsection{Synchronization}\label{synchronization}

Nature exhibits many different kinds of behaviors that we have organized
into different concepts (i.e., random, chaotic, deterministic,
synchronized). THese behaviors have interesting properties and allow us
to bring different and new forms of mathematics and computing to the
work; this includes new areas like
\href{https://inv.tux.pizza/watch?v=2Bw5f4vYL98}{machine-learning based
physics models}. As we move from a single object models to multiple
objects models, we begin to explore these different conceptual
behaviors.
\href{https://www.quantamagazine.org/physicists-discover-exotic-patterns-of-synchronization-20190404/}{Synchronization}
is one that is common and natural to consider in the context of
oscillations as you saw from the above video.
\href{https://inv.tux.pizza/watch?v=ZGvtnE1Wy6U}{Fireflies},
\href{https://inv.tux.pizza/watch?v=2LPboySOSvo}{patterns in the
operation of a heart}, and
\href{https://inv.tux.pizza/watch?v=5v5eBf2KwF8}{coupled metronomes} all
exhibit this synchronized behavior. But, how is that possible? What
physics is going on here? And how can it lead in and out of
synchronization like this pendulum setup below?

\href{https://inv.tux.pizza/watch?v=yVkdfJ9PkRQ}{\pandocbounded{\includegraphics[keepaspectratio,alt={Synchronized pendulum}]{https://markdown-videos-api.jorgenkh.no/youtube/yVkdfJ9PkRQ?width=720&height=405}}}

\begin{itemize}
\tightlist
\item
  Non-Commercial Link: \url{https://inv.tux.pizza/watch?v=yVkdfJ9PkRQ}
\item
  Commercial Link: \url{https://youtube.com/watch?v=yVkdfJ9PkRQ}
\end{itemize}

```\{admonition\} Huygen's Clocks Much of this early understanding of
synchronization we owe to the Dutch mathematician,
\href{https://en.wikipedia.org/wiki/Christiaan_Huygens}{Christiann
Huygens}, who spent much of his work trying to understand periodic
motion. In this work, he observed the oscillation of two pendulum clocks
that would eventually oscillate in precisely opposite swings. A great
writeup of this history and its implications is
\href{https://www.quantamagazine.org/physicists-discover-exotic-patterns-of-synchronization-20190404/}{here}.
This phenomenon was explored experimentally and theoretically by three
physicists at Georgia Tech, Matt Bennet, Mike Schatz (my Phd Advisor!),
and Kurt Wiesenfeld. Their paper was written in collaboration with,
history and language professor emerita, Heidi Rockwood. It's a
\href{http://engineering.nyu.edu/mechatronics/Control_Lab/bck/VKapila/Chaotic\%20Ref/Hujgens.pdf}{fun
read}.

We know use this phenomenon to
\href{https://en.wikipedia.org/wiki/Injection_locking\#Entrainment}{entrain
oscillators}; this kind of mode locking is really important to science
-- especially ultra-fast science. Physicists like
\href{https://www.ogilviegroup.org/}{Dr.~Jennifer Ogilive at the
University of Michigan} use this phenomenon to image biological
materials that undergo very fast transitions. ```

\subsection{Coupled Oscillations}\label{coupled-oscillations}

The key to why these phenomena are possible is the increase in the
number of \textbf{Degrees of Freedom}. You can think of this as the
\href{https://en.wikipedia.org/wiki/Degrees_of_freedom_(mechanics)}{ways
in which the system can move} -- really we should try to think of it as
the
\href{https://en.wikipedia.org/wiki/Degrees_of_freedom_(physics_and_chemistry)}{additional
variables in phase space} that become available by adding more particles
and interactions. We will start with the canonical longitudinal
oscillations of two coupled spring oscillators.

Our analysis will proceed with an investigation of
\href{https://en.wikipedia.org/wiki/Normal_mode}{normal modes}, which is
a powerful tool that lets us not only analyze systems with many degrees
of freedom; it can help us conceptually think about building up
solutions for continuous systems that have infinite degrees of freedom
(i.e., a rope or a water wave).

We start by writing the differential equations that describe the motion
of the system, noting that the length measures are critical for
establishing the right differential equation:

Left mass:

\[m \ddot{x}_{1} = -k x_1 + k'(x_2-x_1)\]

Right mass:

\[m \ddot{x}_{2} = - k'(x_2-x_1)-kx_2\]

We can write these in the following way:

\[\ddot{x}_{1} = -\dfrac{k}{m} x_1 + \dfrac{k'}{m}(x_2-x_1)\]

\[\ddot{x}_{2} = - \dfrac{k'}{m}(x_2-x_1)-\dfrac{k}{m}x_2\]

And then:

\[\ddot{x}_{1} = -\left(\dfrac{k}{m}+\dfrac{k'}{m}\right) x_1 + \dfrac{k'}{m}x_2\]

\[\ddot{x}_{2} = \dfrac{k'}{m}x_1 - \left(\dfrac{k}{m}+\dfrac{k'}{m}\right)x_2\]

This is a linear differential equation. That means we can represent it
as a vector equation:

\[\ddot{\mathbf{x}} = \pmb{A} \mathbf{x}\]

where \(\pmb{A}\) will represent the different coefficients in our
differential equation. Below we write the matrix representation of the
same equation above:

\[\begin{bmatrix}
\ddot{x}_1\\
\ddot{x}_2
\end{bmatrix} = \begin{bmatrix}
-\left(\dfrac{k}{m}+\dfrac{k'}{m}\right) & +\dfrac{k'}{m}\\
+\dfrac{k'}{m} & -\left(\dfrac{k}{m}+\dfrac{k'}{m}\right)
\end{bmatrix}
\begin{bmatrix}
{x}_1\\
{x}_2
\end{bmatrix}
\]

When all the springs are the same \(k\), you will show in class this
leads to two normal modes with frequencies:

\[\omega_1^2 = \dfrac{k}{m}\] \[\omega_2^2 = \dfrac{3k}{m}\]

\subsection{Additional Resources}\label{additional-resources}

This reading is useful preparation for our normal modes discussion.

\subsubsection{Crawford's Waves book - Secs.
1.1-1.5}\label{crawfords-waves-book---secs.-1.1-1.5}

This book is a great resource for learning about oscillations and waves.
Unfortunately, it is out of print. We will discuss a lot of these
examples in class. -
\href{../assets/scans/Crawford-Waves-Secs_1.1-1.3.pdf}{Ch 1.1-1.3 - A
nice intro to waves, with examples and terminology defined} -
\href{../assets/scans/Crawford-Waves-Secs_1.4-1.5.pdf}{Ch 1.4-1.5 -
Normal modes and beats}

These readings will introduce the concept of normal modes, which we will
discuss with a short lecture and example that you will work. And it will
show a phenomenon that we can explore:
\href{https://en.wikipedia.org/wiki/Beat_(acoustics)}{Beats}

\subsubsection{Handwritten Notes}\label{handwritten-notes}

Here are my handwritten notes on coupled oscillations and normal modes.

\begin{itemize}
\tightlist
\item
  \href{../assets/notes/Notes-Coupled_Oscillations.pdf}{Coupled
  Oscillations}
\item
  \href{../assets/notes/Notes-Three_Coupled_Oscillators.pdf}{Three
  coupled oscillator example}
\end{itemize}

\subsubsection{Video Resources}\label{video-resources}

If you are a feeling that you would like a little more on oscillations,
beats, and matrices before we have class. Check out this video.

\href{https://inv.tux.pizza/watch?v=I0YACDaY-ww}{\pandocbounded{\includegraphics[keepaspectratio,alt={Coupled Oscillator Lecture}]{https://markdown-videos-api.jorgenkh.no/youtube/I0YACDaY-ww?width=720&height=405}}}

\begin{itemize}
\tightlist
\item
  Non-Commercial Link: \url{https://inv.tux.pizza/watch?v=I0YACDaY-ww}
\item
  Commercial Link: \url{https://youtube.com/watch?v=I0YACDaY-ww}
\end{itemize}
