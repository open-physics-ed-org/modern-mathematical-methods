\section{28 Sep 23 - Activity: Superposition of
Fields}\label{sep-23---activity-superposition-of-fields}

\subsection{Coulomb's Law}\label{coulombs-law}

We've seen that the electric field of a set of point charges if given by
the vector sum of the contributions from each charge:

\[\mathbf{E}(\mathbf{r}) = \sum_i \frac{1}{4\pi\epsilon_0} \frac{q_i}{(\mathbf{r}-\mathbf{r}'_i)^3} (\mathbf{r}-\mathbf{r}'_i)\]

where \(\mathbf{r}'_i\) is the location of the charge relative to the
origin, and \(\mathbf{r}\) is the location of the field point relative
to the origin. This \textbf{net electric field} is the result of the
\textbf{superposition} of the electric fields from each charge. But here
the charges are discrete, we can identify each charge uniquely in terms
of it's charge and location. This concept can be abstracted to a
continuous distribution of charge, where we can identify the charge at
any point in space by the charge density \(\rho(\mathbf{r})\). In this
case, the charge is not a point charge, but rather a small chunk of
charge that we model as a point charge. That is, we can write the
electric field contributed by that chunk \(dq\) as:

\[d\mathbf{E}(\mathbf{r}) = \frac{1}{4\pi\epsilon_0} \frac{dq}{(\mathbf{r}-\mathbf{r}')^3} (\mathbf{r}-\mathbf{r}')\]

where we have treated the chunk as a point charge. In the generic case,
that chunk is given by \(dq = \rho(\mathbf{r})dV\) so that,

\[d\mathbf{E}(\mathbf{r}) = \frac{1}{4\pi\epsilon_0} \frac{\rho(\mathbf{r})dV}{(\mathbf{r}-\mathbf{r}')^3} (\mathbf{r}-\mathbf{r}')\]

The net electric field is then given by the integral over the entire
charge distribution:

\[\mathbf{E}(\mathbf{r}) = \iiint \frac{1}{4\pi\epsilon_0} \frac{\rho(\mathbf{r}')dV}{(\mathbf{r}-\mathbf{r}')^3} (\mathbf{r}-\mathbf{r}')\]

This gives a general prescription for finding the electric field in most
situations (i.e., where the integral converges). But this appraoch is
only analytically tractable in situations where we are able to find
anti-derivatives of the integrands that we construct. There are many,
many techniques to evaluating tricky integrals. But in the end, we often
turn to technologies like Mathematica,
\href{https://www.wolframalpha.com/}{Wolfram Alpha}, or
\href{https://www.sagemath.org/}{SageMath} to do the heavy lifting for
us. Numerical methods are also very useful, and we already used a solver
for 2D problems.

\subsubsection{Activity: Ring of Charge}\label{activity-ring-of-charge}

Consider the ring (radius, \(R\)) with a uniform linear charge density
\(\lambda = \frac{Q}{2\pi R}\). that is placed with it's center on the
plane as shown below.

\begin{figure}
\centering
\pandocbounded{\includegraphics[keepaspectratio,alt={Ring of charge diagram}]{https://blog.cupcakephysics.com/assets/images/2015-06-21/ring.png}}
\caption{Ring of charge diagram}
\end{figure}

\textbf{✅ Do this}

\begin{enumerate}
\def\labelenumi{\arabic{enumi}.}
\tightlist
\item
  Write an expression for the contribution of an arbitrary charge
  element to the electric field at the location shown
  \(\mathbf{r} = \langle 0,0,z \rangle\). What is charge density and
  what is the charge element?
\item
  Rewrite the expression as an integral over the relevant charge
  distribution. What are the limits of integration?
\item
  (if you want) Use a symbolic integrator to evaluate the integral. What
  is the electric field at the location shown? (You can use
  \href{https://www.wolframalpha.com/}{Wolfram Alpha} or try Python's
  own symbolic toolkit
  \href{https://www.sympy.org/en/index.html}{SymPy})
\end{enumerate}

\begin{Shaded}
\begin{Highlighting}[]
\ImportTok{import}\NormalTok{ numpy }\ImportTok{as}\NormalTok{ np}
\ImportTok{import}\NormalTok{ matplotlib.pyplot }\ImportTok{as}\NormalTok{ plt}
\CommentTok{\#\# your code to plot the electric field expression}
\end{Highlighting}
\end{Shaded}

\subsection{Magnetostatic Analogs}\label{magnetostatic-analogs}

As we discussed earlier, a magnetostatic situation is described using
the following PDEs:

\[\nabla \cdot \mathbf{B} = 0\]

\[\nabla \times \mathbf{B} = \mu_0 \mathbf{J}\]

where \(\mathbf{J}\) is the current density.

\subsubsection{Biot-Savart Law}\label{biot-savart-law}

There is a solution to these equations that is analogous to the
Coulomb's Law solution for the electric field. This solution is called
the
\textbf{\href{https://en.wikipedia.org/wiki/Biot\%E2\%80\%93Savart_law}{Biot-Savart
Law}}. The contribution of a little chunk of current at a location
\(\mathbf{r}'\) to the magnetic field at a location \(\mathbf{r}\) is
given by:

\[d\mathbf{B}(\mathbf{r}) = \frac{\mu_0}{4\pi} \frac{\mathbf{J} dV \times (\mathbf{r}-\mathbf{r}')}{(\mathbf{r}-\mathbf{r}')^3}\]

where \(I\) is the current in the chunk of wire \(d\mathbf{l}\) and
\(\mathbf{r}-\mathbf{r}'\) is the vector from the chunk to the field
point. The net magnetic field is then given by the integral over the
entire current distribution:

\[\mathbf{B}(\mathbf{r}) = \frac{\mu_0}{4\pi} \iiint \frac{\mathbf{J} dV \times (\mathbf{r}-\mathbf{r}')}{(\mathbf{r}-\mathbf{r}')^3}\]

This full form of the Biot-Savart Law suffers from the some of the same
challenges that we see with the full form of Coloumb's Law. It can only
be computed directly if we can form integrals that we can perform
analytically. Otherwise it can form the basis for other approaches.
Typically, we use the Biot-Savart Law to compute the magnetic field for
a steady current and for geometries that have a high degree of symmetry.
The most common case is a thin wire with a steady, constant current
\(I\). Then the full form simplifies to the line integral:

\[\mathbf{B}(\mathbf{r}) = \frac{\mu_0 I}{4\pi} \int \frac{d\mathbf{l} \times (\mathbf{r}-\mathbf{r}')}{(\mathbf{r}-\mathbf{r}')^3}\]

where \(d\mathbf{l}\) is the vector element of the wire. This is the
form that you are likely to use most often.

\subsubsection{Ampere's Law}\label{amperes-law}

Ampere's Law uses the curl equation for \(\mathbf{B}\) and Stokes'
theorem to produce a line integral that can be used to solve for the
magnetic field. We can derive this quickly:

\[\iint \nabla \times \mathbf{B} \cdot d\mathbf{A} = \mu_0 \iint \mathbf{J} \cdot d\mathbf{A}\]

\[\oint \mathbf{B} \cdot d\mathbf{l} = \mu_0 \iint \mathbf{J} \cdot d\mathbf{A} = \mu_0 I_{enc}\]

where \(I_{enc}\) is the current enclosed by the path of integration.
This integral is always true, but like Gauss'\,'s Law, it is not useful
unless there's a high degree of symmetry.

\paragraph{A thin wire}\label{a-thin-wire}

The example of a thin wire is common and forms the basis for more
complex forms of Ampere's Law. Consider a thin wire with a steady
current \(I\). We are a distance \(r\) from the wire. We know the
magnetic field circulates around the wire, so we choose a circular path
a distance \(r\) from the wire.

\begin{figure}
\centering
\pandocbounded{\includegraphics[keepaspectratio,alt={Thin wire}]{../images/activity-superposition_ampere_loop.jpg}}
\caption{Thin wire}
\end{figure}

In this case the (as of yet uncomputed) magnetic field is always
parallel to out choice of path. Moerover, we have no reason to believe
the magnetic field strength is different at different locations along
the circle. So the integral simplifies:

\[\oint \mathbf{B} \cdot d\mathbf{l} = \oint B dl = B \oint dl = B 2\pi r\]

where \(B\) is the magnitude of the magnetic field. The total current is
just \(I\) so that,

\[B 2\pi r = \mu_0 I\]

\[\mathbf{B}({\mathbf{r}}) = \frac{\mu_0 I}{2\pi r}\hat{\phi}\]

where \(\hat{\phi}\) is the unit vector in the azimuthal direction --
around the wire.

\subsubsection{A thick wire}\label{a-thick-wire}

Consider a thick wire (radius \(a\)) that has a steady current density
\(\mathbf{J}(\mathbf{r}) = J_0 \hat{z}\) where \(J_0\) is some constant.
See the figure below.

\begin{figure}
\centering
\pandocbounded{\includegraphics[keepaspectratio,alt={Thick wire}]{../images/activity-superposition_ampere-thick.png}}
\caption{Thick wire}
\end{figure}

\textbf{✅ Do this}

\begin{enumerate}
\def\labelenumi{\arabic{enumi}.}
\tightlist
\item
  Use Ampere's Law to find the magnetic field inside and outside the
  wire.
\item
  Change the current density to
  \(\mathbf{J}(\mathbf{r}) = J_0 \frac{r^2}{a^2} \hat{z}\) and find the
  field again. How do the fields outside compare between case
  \(\mathbf{J}(\mathbf{r}) = J_0 \hat{z}\) and this one.
\item
  How could you write the current density so that it only appears on the
  surface of the thick wire? (Hint: look into
  \href{https://en.wikipedia.org/wiki/Dirac_delta_function}{Dirac delta
  functions}.)
\item
  Can you use Ampere's Law if the current density in the wire were
  \$\mathbf{J}(\mathbf{r}) = \sigma\_0 \delta(r-a) \hat{z} \$? Why or
  why not? What does that current distribution look like?
\item
  Can you use Ampere's Law if the current density in the wire were
  \(\mathbf{J}(\mathbf{r}) = J_0\frac{r}{a} \hat{\phi}\)? Why or why
  not? What does that current distribution look like?
\item
  Can you use Ampere's Law if the current density in the wire were
  \(\mathbf{J}(\mathbf{r}) = \sigma_0 \delta(r-a) \hat{\phi}\)? Why or
  why not? What does that current distribution look like?
\end{enumerate}
