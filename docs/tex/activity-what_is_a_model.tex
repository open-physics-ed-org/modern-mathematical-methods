\section{29 Aug 23 - Activity: What is a
Model?}\label{aug-23---activity-what-is-a-model}

\subsection{Notes}\label{notes}

\begin{itemize}
\tightlist
\item
  \emph{Class Meeting:} 29 Sept 23
\item
  \emph{Prior Reading:} None necessary
\end{itemize}

\subsection{What is Mathematical
Modeling?}\label{what-is-mathematical-modeling}

Nature reveals itself to us through interactions. We can tell from
observations that it is nature's interactions that lead to its
evolution. How nature is changing and predicting how it will change in
the future is the work of science. In this work, we observe nature and
its interactions to make models of those observations. We aim to predict
and explain our observations of nature through this building of models.

In physics, our goals are typically to explain and predict observations
of physical phenomenon. Here, we focus ourselves to those canonical
things that physicists concern themselves with: motion, fields, waves,
atoms, nuclei, and so on.

\subsection{Models and modeling}\label{models-and-modeling}

This course focuses on common physical models and how we use mathematics
and computing tools to investigate these models. But, we need to agree
on an mutual understanding of a model. In this activity, you will be
working in a group to define aspects of a good model for the physical
sciences.

There are plenty of ideas on the internet, I can even
\href{https://bfy.tw/TUdA}{provide you a link} with plenty of
information. But, the point of this activity is for us to define a good
model now (given all your prior experiences with models and modeling),
and to reflect on that definition as you develop expertise with models
and modeling (through your learning in this class).

\subsection{What is a model?}\label{what-is-a-model}

We are going to start with this short video made by Geoscientist John
Aiken when he was a graduate student at Georgia Tech.

\href{https://inv.tux.pizza/watch?v=dkTncoPqo5Y}{\pandocbounded{\includegraphics[keepaspectratio]{https://markdown-videos-api.jorgenkh.no/youtube/dkTncoPqo5Y?width=720&height=405}}}
- Non-Commercial Link: \url{https://inv.tux.pizza/watch?v=dkTncoPqo5Y} -
Commercial Link: \url{https://youtube.com/watch?v=dkTncoPqo5Y}

\subsubsection{History and Philosophy of
Science}\label{history-and-philosophy-of-science}

If you would like to dive deeper into models and modeling, there's
excellent work in history and philosophy of science. The field studies
how science develops knowledge, practice, culture, and so on. It studies
important events and provides critical information on important and,
often, overlooked folks who do science. For example, historian and
gender studies professor
\href{https://en.wikipedia.org/wiki/Sharon_Traweek}{Sharon Traweek}
studies the high energy physics field. Her book,
\href{https://en.wikipedia.org/wiki/Beamtimes_and_Lifetimes}{Beamtimes
and Lifetimes: The World of High Energy Physicists}
\{cite\}\texttt{traweek2009beamtimes} is excellent.

```\{admonition\} Dame Nancy Cartwright (philosopher of science) One of
the more interesting scholars in
\href{https://en.wikipedia.org/wiki/Nancy_Cartwright_(philosopher)}{Dame
Nancy Cartwright} who wrote a lot about the `practice of science.' Her
philosophical work informed many of the innovations in physics and
broader science education -- including many science courses at MSU.

Her writing is very interesting, but the style of writing can be a
challenge to read. This is the nature of academic writing in different
disciplines. Her book called ``How The Laws of Physics Lie''
\{cite\}\texttt{cartwright1983laws} is worth a read. Here's a link to
the
\href{http://www.generativescience.org/papers/nature/Cartwright-_1983.pdf}{first
chapter}.

\begin{verbatim}

## What has been your experience with models?

Take 2-3 minutes to think about your prior physics class.
- What models have you used? What makes that a model?
- What modeling have you done? What makes that modeling?


```python
## Take notes for yourself here

# Idea 1
# Idea 2
\end{verbatim}

After we have discussed the above questions and generated a long list of
experiences, you can move onto the next set of questions below.

\texttt{\{toggle\}\ Take\ 2-3\ minutes\ to\ think\ about\ your\ prior\ work\ with\ models\ and\ modeling?\ -\ What\ made\ a\ model\ good\ or\ not\ so\ good?\ -\ What\ kinds\ of\ things\ could\ you\ do\ to\ make\ a\ better\ model?}

\begin{Shaded}
\begin{Highlighting}[]
\CommentTok{\#\# Take notes for yourself here}

\CommentTok{\# Idea 1}
\CommentTok{\# Idea 2}
\end{Highlighting}
\end{Shaded}

After we have discussed the above questions and generated a consensus
list of good model features, you can move onto the next set of questions
below.

\texttt{\{toggle\}\ Take\ 2-3\ minutes\ and\ write\ down\ all\ the\ mathematical\ and/or\ computing\ models\ you\ have\ experienced,\ heard\ about,\ seen,\ or\ are\ otherwise\ familiar.}

\begin{Shaded}
\begin{Highlighting}[]
\CommentTok{\#\# Take notes for yourself here (use comments)}

\CommentTok{\# Model 1}
\CommentTok{\# Model 2}
\end{Highlighting}
\end{Shaded}
