\section{4 Dec 23 - Notes: Markov Chain Monte Carlo
Modeling}\label{dec-23---notes-markov-chain-monte-carlo-modeling}

One of the principal tools we have at our disposal when it comes to
systems with randomness is the Monte Carlo simulation. Monte Carlo is a
large class of approaches that all rely on probabilistic outcomes to
determine results. In physics, we often use Monte Carlo to find
integrals or sums that we are unable to compute by hand.

\subsection{Markov Chain Monte Carlo}\label{markov-chain-monte-carlo}

We often find ourselves using a type of Monte Carlo simulation that
makes use of a Markov Chain. This is called the MCMC model. The
mathematical foundations of Markov Chains are well beyond the scope of
this course, but the principal issue is that we want to describe is how
this simulation works. The video below describes the conceptual backing
of MCMC and how it was first used in physics in 1953 to develop a model
of an ideal gas. We will make a similar model!

\href{https://inv.tux.pizza/watch?v=12eZWG0Z5gY}{\pandocbounded{\includegraphics[keepaspectratio]{https://markdown-videos-api.jorgenkh.no/youtube/12eZWG0Z5gY?width=720&height=405}}}

\begin{itemize}
\tightlist
\item
  Non-Commercial Link: \url{https://inv.tux.pizza/watch?v=12eZWG0Z5gY}
\item
  Commercial Link: \url{https://youtube.com/watch?v=12eZWG0Z5gY}
\end{itemize}

\subsection{Statistical Mechanics}\label{statistical-mechanics}

MCMC is used in lots of statistical mechanics problems because they are
fundamentally probabilistic by nature. Starting with the ``chance'' of
finding our system in a given state (with known energy, \(E_i\)) at a
known temperature (T) given by the Boltzmann factor:

\[e^{-E_i/{k_b T}}\]

where \(k_b\) is the Boltzmann constant. Through this, we developed a
statistical model where we found that the normalized probability of
finding your system in a state \(s\) with energy \(E_s\) (just using the
\(s\) to indicate a state) is given by:

\[P(s) = \dfrac{1}{Z} e^{-E_s/{k_b T}}\]

where \(Z\) is the partition function, a constant for a given
temperature that normalizes our calculation. It is a sum over all
states:

\[Z = \sum_s e^{-E_s/{k_b T}}\]

Our analysis relied on the development of a statistical theory of
mechanics, and we illustrated it with an ideal gas.

Because \(P(S)\) is a probability we can use it find average values
(expectation values) of a thermodynamic system. We did this for energy
using the thermodynamic relations in the notes. But we can also use
statistical properties to find the same. For example, finding the
expected internal energy of a system, \(\langle U \rangle\), just
involves adding up all the possible energy states multiplied by their
probabilities!

\[\langle U \rangle = \sum_s E_s P(s)\]

When these sums are really hard to compute because there's lots of
states or only a few that contribute substantially (as in the case for
large systems), we can use selective sampling, which is the basis for
MCMC. We will discuss that conceptually in class before using MCMC.

\subsection{Additional Resources}\label{additional-resources}

\subsubsection{Handwritten Notes}\label{handwritten-notes}

\begin{itemize}
\tightlist
\item
  \href{../assets/notes/Notes-Intro_to_Stat_Mech.pdf}{Introduction to
  Stat. Mech.}
\item
  \href{../assets/notes/Notes-Markov_Chain.pdf}{Markvov Chain Monte
  Carlo}
\end{itemize}

\subsubsection{Lecture Videos}\label{lecture-videos}

\textbf{MCMC for Data Science}

\href{https://inv.tux.pizza/watch?v=yApmR-c_hKU}{\pandocbounded{\includegraphics[keepaspectratio]{https://markdown-videos-api.jorgenkh.no/youtube/yApmR-c_hKU?width=720&height=405}}}

\begin{itemize}
\tightlist
\item
  Non-Commercial Link: \url{https://inv.tux.pizza/watch?v=yApmR-c_hKU}
\item
  Commercial Link: \url{https://youtube.com/watch?v=yApmR-c_hKU}
\end{itemize}

\textbf{Details on the MCMC Algorithm}

\href{https://inv.tux.pizza/watch?v=rZk2FqX2XnY}{\pandocbounded{\includegraphics[keepaspectratio]{https://markdown-videos-api.jorgenkh.no/youtube/rZk2FqX2XnY?width=720&height=405}}}

\begin{itemize}
\tightlist
\item
  Non-Commercial Link: \url{https://inv.tux.pizza/watch?v=rZk2FqX2XnY}
\item
  Commercial Link: \url{https://youtube.com/watch?v=rZk2FqX2XnY}
\end{itemize}
