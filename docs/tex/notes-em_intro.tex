\section{26 Sept 23 - Notes: Maxwell's
Equations}\label{sept-23---notes-maxwells-equations}

\subsection{Electromagnetic Phenomena}\label{electromagnetic-phenomena}

You have certainly experienced a wide variety of electromagnetic
phenomena in your life -- both natural (e.g., a sun rise, a rainbow,
your hair standing on end) and human-made (e.g., switching on a light,
using a microwave, using a cell phone). We now know that we can explain
many of those phenomenon using a set of four partial differential
equations:
\href{https://en.wikipedia.org/wiki/Maxwell\%27s_equations}{Maxwell's
Equations}. These equations describe a new mathematical object: a
\href{https://en.wikipedia.org/wiki/Field_(physics)}{field}. The models
that we develop from this theory are still
\href{https://en.wikipedia.org/wiki/Classical_physics}{Classical} -- in
that we don't consider quantum mechanical effects. Rather we treat the
fields as continuous functions of space and time.

While this might not seem that this classical theory is still useful in
modern research, it is often the basis for much of the work done in
physics. In fact, some important fusion energy research uses
\href{https://en.wikipedia.org/wiki/Magnetic_confinement_fusion}{magnetic
confinement}, which is effectively a classical electromagnetic
phenomenon. The South Korean experimental fusion reactor,
\href{https://en.wikipedia.org/wiki/KSTAR}{KSTAR}, is an example of this
type of research, and it recently
\href{https://www.popularmechanics.com/science/energy/a41191247/koreas-fusion-reactor-sustained-temperatures-7-times-hotter-than-the-sun-for-30-seconds/}{sustained
temperatures of 100 million degrees Celsius}.

\subsubsection{Magnetic Confinement in Fusion
Energy}\label{magnetic-confinement-in-fusion-energy}

\href{https://inv.tux.pizza/watch?v=PWCqwZoE0FY}{\pandocbounded{\includegraphics[keepaspectratio,alt={Magnetic Confinement}]{https://markdown-videos-api.jorgenkh.no/youtube/PWCqwZoE0FY?width=720&height=405}}}

\begin{itemize}
\tightlist
\item
  Non-Commercial Link: \url{https://inv.tux.pizza/watch?v=PWCqwZoE0FY}
\item
  Commercial Link: \url{https://youtube.com/watch?v=PWCqwZoE0FY}
\end{itemize}

\subsection{Developing a complete
theory}\label{developing-a-complete-theory}

The work to develop
\href{https://en.wikipedia.org/wiki/Maxwell\%27s_equations}{Maxwell's
Equations} was substantial and involved not only theoretical
developments but confirmations using experiments. A model of reality is
only as good as its ability to predict the results of experiments, which
classical electromagnetism does very well (up to the point of quantum
phenomena). The resulting Maxwell Equations that describe the electric
(\(\mathbf{E}\)) and magnetic (\(\mathbf{B}\)) fields (in vacuum) are
given by:

\[
\textbf{Differential Form:}
\]

Gauss's Law for Electricity:

\[
\mathbf{\nabla} \cdot \mathbf{E} = \frac{\rho}{\varepsilon_0}
\]

Gauss's Law for Magnetism:

\[
\mathbf{\nabla} \cdot \mathbf{B} = 0
\]

Faraday's Law of Electromagnetic Induction:

\[
\mathbf{\nabla} \times \mathbf{E} = -\frac{\partial \mathbf{B}}{\partial t}
\]

Ampère's Law with Maxwell's Addition:

\[
\mathbf{\nabla} \times \mathbf{B} = \mu_0 \mathbf{J} + \mu_0 \varepsilon_0 \frac{\partial \mathbf{E}}{\partial t}
\]

\[
\textbf{Integral Form:}
\]

Gauss's Law for Electricity:

\[
\oint \mathbf{E} \cdot d\mathbf{A} = \frac{1}{\varepsilon_0} \int \rho dV
\]

Gauss's Law for Magnetism:

\[
\oint \mathbf{B} \cdot d\mathbf{A} = 0
\]

Faraday's Law of Electromagnetic Induction:

\[
\oint \mathbf{E} \cdot d\mathbf{l} = -\frac{d}{dt} \int \mathbf{B} \cdot d\mathbf{A}
\]

Ampère's Law with Maxwell's Addition:

\[
\oint \mathbf{B} \cdot d\mathbf{l} = \mu_0 \int \mathbf{J} \cdot d\mathbf{A} + \mu_0 \varepsilon_0 \frac{d}{dt} \int \mathbf{E} \cdot d\mathbf{A}
\]

We will work with different aspects of these equations as we make
different assumptions about the physical system.

\subsubsection{A comment on the development of this
theory}\label{a-comment-on-the-development-of-this-theory}

During the development of Classical Electromagnetism, physics research
was often lead by single individuals with small armies of technicians,
specialists, drafters, secretarial staff, and the like. While presented
as the intellectual work of a single person, the work was an effort by a
number of underpaid and under-recognized people (including, at the time,
women and non-white men) who are unnamed in many books. A classical, and
non critical, view of the development appears below in the video, but
also in Meyer's book
\href{https://mitpress.mit.edu/9780262130707/a-history-of-electricity-and-magnetism/}{A
History of Electricity and Magnetism}. A more nuanced view of how
science develops and is negotiated appears in Traweek's book
\href{https://en.wikipedia.org/wiki/Beamtimes_and_Lifetimes}{Beamtimes
and Lifetimes}.

\paragraph{The Mechanical Universe}\label{the-mechanical-universe}

This video from the
\href{https://en.wikipedia.org/wiki/The_Mechanical_Universe}{Mechanical
Universe series} and tells the story of the development of this theory.
The video is dated and sexist and the instructor's stories are similarly
problematic. The video puts scientists on a pedestal. But, it is worth a
watch, both to recognize the history as presented and to reckon with the
issues that these perspectives have continued to perpetuate.

\href{https://inv.tux.pizza/watch?v=SS4tcajTsW8}{\pandocbounded{\includegraphics[keepaspectratio,alt={Maxwell's Equations}]{https://markdown-videos-api.jorgenkh.no/youtube/SS4tcajTsW8?width=720&height=405}}}

\begin{itemize}
\tightlist
\item
  Non-Commercial Link: \url{https://inv.tux.pizza/watch?v=SS4tcajTsW8}
\item
  Commercial Link: \url{https://youtube.com/watch?v=SS4tcajTsW8}
\end{itemize}
