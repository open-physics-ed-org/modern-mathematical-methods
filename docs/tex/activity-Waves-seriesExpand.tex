\section{2 Nov 23 - Activity: Automated Signal Deconstruction \&
Reconstruction}\label{nov-23---activity-automated-signal-deconstruction-reconstruction}

We found that it was possible to use the Fourier transform to decompose
the signal into its constituent frequencies. Later we will use this idea
to build up the
\href{https://en.wikipedia.org/wiki/Fast_Fourier_transform}{Fast Fourier
Transform} (FFT) algorithm. For now, we will write everything out in
full to mirror our analytical work.

We will start with the same signal as before, but then you will work
with other signals to achieve a particular signal decomposition.

Recall that the sign was given by the square wave function:

\[V(t) = \begin{cases} 0 & \text{if } 0 \leq t \leq \frac{T}{2} \\ V_0 & \text{if } \frac{T}{2} < t \leq T \end{cases}\]

We define the approximate form of the signal using the complex form of
the Fourier series:

\[V(t) = \sum_{-\infty}^{+\infty} c_n e^{i n \omega_0 t}\]

where \(\omega_0 = \frac{2 \pi}{T_0}\) and \(c_n\) is the Fourier
coefficient given by:

\[c_n = \frac{1}{T_0} \int_{0}^{T_0} V(t) e^{-i n \omega_0 t} dt\]

Remember that we use the real \(V(t)\) to compute the complex \(c_n\).
And once we have found the \(c_n\) we can reconstruct the signal using
the series expansion.

\[V(t) \approx V_{approx}(t) = \sum_{-N}^{+N} c_n e^{i n \omega_0 t}\]

where \(N\) is the number of terms we use in the series expansion.

\subsection{Analytical Calculation}\label{analytical-calculation}

\textbf{✅ Do this}

\begin{enumerate}
\def\labelenumi{\arabic{enumi}.}
\tightlist
\item
  Take the signal \(V(t)\) and determine which integrals you have to do.
\item
  Construct the integrals you need to do (over what cycle are you
  integrating?).
\item
  Perform the integrals (consider using \texttt{sympy}) and find the
  coefficients \(c_n\)'s.
\item
  Write the function \(V(t)\) in terms of the \(c_n\)'s. (What about the
  imaginary part?)
\end{enumerate}

\subsection{Automating the
Calculation}\label{automating-the-calculation}

This is a lot of work to do by hand, and we have to perform new
integrals every time we want to find this decomposition. We can automate
this process using a little code. We have to break the process up into a
few steps.

\begin{enumerate}
\def\labelenumi{\arabic{enumi}.}
\tightlist
\item
  Define the signal \(V(t)\).

  \begin{itemize}
  \tightlist
  \item
    Here you need to make sure you have a function that can be evaluated
    for any \(t\).
  \item
    You should also be able to change that function easily for different
    signals.
  \end{itemize}
\item
  Compute the Fourier coefficient \(c_n\) for a given \(n\).

  \begin{itemize}
  \tightlist
  \item
    Here you need to perform an integral of \(V(t)e^{-i n \omega_0 t}\)
    over the period \(T_0\).\\
  \item
    It would be good to use a builtin integrator like
    \texttt{numpy.trapz} or \texttt{scipy.integrate.quad} to do this.
  \item
    The documentation for numpy.trapz can be found
    \href{https://docs.scipy.org/doc/numpy/reference/generated/numpy.trapz.html}{here}
    and the documentation for scipy.integrate.quad can be found
    \href{https://docs.scipy.org/doc/scipy/reference/generated/scipy.integrate.quad.html}{here}.
  \end{itemize}
\item
  Compute the approximation \(V_{approx}(t)\) up to a given \(N\).

  \begin{itemize}
  \tightlist
  \item
    Here you need a complex array to take on the approximate values for
    \(V(t)\).
  \item
    You will need to loop over the \(n\) values and add the contribution
    from each \(c_n\) to the array.
  \item
    Return the real part of the array to get the approximation.
  \end{itemize}
\end{enumerate}

\textbf{✅ Do this}

\begin{enumerate}
\def\labelenumi{\arabic{enumi}.}
\tightlist
\item
  Review the code scaffold below and make sure you understand what is
  happening.
\item
  Complete the code below to compute the Fourier coefficients \(c_n\)
  for a given \(n\).
\item
  Complete the code below to compute the approximation \(V_{approx}(t)\)
  up to a given \(N\).
\item
  Test your code with the provided signal and plot the approximation for
  different values of \(N\).
\item
  Write function that plots your real and approximate signal on the same
  plot.
\item
  Write a function that plots the magnitude of Fourier coefficients
  \(c_n\) as a function of \(n\omega_0\).
\end{enumerate}

\begin{Shaded}
\begin{Highlighting}[]
\ImportTok{import}\NormalTok{ numpy }\ImportTok{as}\NormalTok{ np}
\ImportTok{import}\NormalTok{ matplotlib.pyplot }\ImportTok{as}\NormalTok{ plt}
\ImportTok{from}\NormalTok{ scipy.signal }\ImportTok{import}\NormalTok{ square}

\CommentTok{\# Compute the Fourier coefficient}
\KeywordTok{def}\NormalTok{ compute\_cn(v, n, T, f0):}
    
    \CommentTok{\#\# CODE HERE \#\#}
    
    \ControlFlowTok{return}\NormalTok{ cn}

\CommentTok{\# Fourier series expansion using complex form}
\KeywordTok{def}\NormalTok{ complex\_fourier\_series\_expansion(v, t, T, N):}
\NormalTok{    f0 }\OperatorTok{=} \DecValTok{1} \OperatorTok{/}\NormalTok{ T}
\NormalTok{    series }\OperatorTok{=}\NormalTok{ np.zeros\_like(t, dtype}\OperatorTok{=}\BuiltInTok{complex}\NormalTok{)}

    \CommentTok{\#\# CODE HERE \#\#}
        
    \ControlFlowTok{return}\NormalTok{ series.real}

\CommentTok{\# Define the square wave signal}
\NormalTok{T }\OperatorTok{=} \DecValTok{2} \OperatorTok{*}\NormalTok{ np.pi  }\CommentTok{\# Period}
\KeywordTok{def}\NormalTok{ v(t):}
    \ControlFlowTok{return}\NormalTok{ square(t)}
\end{Highlighting}
\end{Shaded}

\subsection{Physics Relationship}\label{physics-relationship}

You might be wondering why we do this kind of deconstruction and what it
is used for. It turns out that for many systems this kind of analysis
tells you how energy, momentum, and other quantities might be
distributed in the system. For example, the analogy to a complex quantum
mechanical system might be the various occupied energy states of the
system. The known frequencies could correspond to the energy levels of
the system. The Fourier coefficients tell you how much of the system is
in each energy state. There are many such analogies in physics; and many
such transformations in multiple dimensions to extract this information.
Fourier is just one of the most common.

\subsection{Decomposition Matching}\label{decomposition-matching}

We want to develop your intuition for this kind of analysis now that you
have started to develop the tools to do it. So we will ask you to use
your code to perform a few tasks. But these will start from the expected
coefficients and ask you to find the signal. This is the reverse of what
we have done so far. With your code you should be able to play and
adjust the signal to match the coefficients qualitatively.

\textbf{✅ Do this}

Use your code to find the signal that matches the following
coefficients: 1. No odd coefficients. 2. No even coefficients. 3. Only
the \(n=1\) coefficient. 4. Only the \(n=2\) coefficient. 5. High values
for low \(n\) and low values for high \(n\) (not too high). 6. Low
values for low \(n\) and high values for high \(n\) (not too high). 7.
Roughly equal values for all \(n\).

In doing this, take note of the signals you are making. What are you
doing to construct them to produce this kind of output. What kind of
device or physical system might produce those signals?

\begin{Shaded}
\begin{Highlighting}[]
\CommentTok{\#\#\# your code here}
\end{Highlighting}
\end{Shaded}
