\section{26 Sept 23 - Notes: Static
Fields}\label{sept-23---notes-static-fields}

\subsection{Vector Calculus}\label{vector-calculus}

\href{https://en.wikipedia.org/wiki/Maxwell\%27s_equations}{Maxwell's
Equations}, even in their static form (i.e., static charges or steady
currents), are a set of coupled partial differential equations. In order
to develop solutions and models to approximate the behavior of
electromagnetic fields, we need to be able to use vector calculus.

The video below from \href{https://www.3blue1brown.com/}{3Blue1Brown}
provides a conceptual introduction to the vector calculus that we are
going to need for this.

\href{https://inv.tux.pizza/watch?v=rB83DpBJQsE}{\pandocbounded{\includegraphics[keepaspectratio,alt={Vector Calculus}]{https://markdown-videos-api.jorgenkh.no/youtube/rB83DpBJQsE?width=720&height=405}}}

\begin{itemize}
\tightlist
\item
  Non-Commercial Link: \url{https://inv.tux.pizza/watch?v=rB83DpBJQsE}
\item
  Commercial Link: \url{https://youtube.com/watch?v=rB83DpBJQsE}
\end{itemize}

\subsection{Electrostatic fields}\label{electrostatic-fields}

The electrostatic field is one that is created by static (stationary)
charges. This is often the first type of field that we encounter, where
\(\rho(\mathbf{r},t)) = \rho(\mathbf{r})\). Because the charges are
stationary, the current density is zero,
\(\mathbf{J}(\mathbf{r},t) = \mathbf{0}\).

\subsubsection{The electric field}\label{the-electric-field}

The relevant Maxwell equations for \(\mathbf{E}\) that govern an
electrostatic situation are the following:

\[\nabla \cdot \mathbf{E} = \frac{\rho}{\varepsilon_0}\]

\[\nabla \times \mathbf{E} = 0\]

You might recognize the first equation as the differential form of
\href{https://en.wikipedia.org/wiki/Gauss\%27s_law}{Gauss's Law}.

\begin{figure}
\centering
\pandocbounded{\includegraphics[keepaspectratio,alt={Gauss's Law}]{https://subratachak.files.wordpress.com/2017/12/gauss-law-diagram-03.jpg?w=782}}
\caption{Gauss's Law}
\end{figure}

As you can see from the figure above, Gauss's Law is always true, but
almost never useful in solving problem outside specific geometries. The
integral form of that equation is:

\[\oint \mathbf{E} \cdot d\mathbf{A} = \frac{Q_{enc}}{\varepsilon_0}\]

It is from considering the integral form of Gauss' Law that we can
obtain the electric field produced by a single point charge, \(Q\), at a
location \(\mathbf{r}'\):

\[\mathbf{E}(\mathbf{r}) = \frac{Q}{4\pi\varepsilon_0} \frac{\mathbf{r} - \mathbf{r}'}{|\mathbf{r} - \mathbf{r}'|^3}\]

The second relevant equation is a statement that the curl of the
electrostatic field is zero. This indicates that an electrostatic field
is a
\href{https://en.wikipedia.org/wiki/Conservative_vector_field}{conservative
field}. This means that the electrostatic field can be written as the
gradient of a scalar potential, \(\mathbf{E} = -\nabla V\), as we will
see later.

There is a general integral solution the electrostatic problem, which is
built up from the concept of a point charge. That solution, which can be
used in certain integrable situations and adapted to numerical methods
is:

\[d\mathbf{E} = \frac{1}{4\pi\varepsilon_0} \frac{dq}{|\mathbf{r} - \mathbf{r}'|^3} (\mathbf{r} - \mathbf{r}')\]

\[\mathbf{E}(\mathbf{r}) = \frac{1}{4\pi\varepsilon_0} \int \frac{dq}{|\mathbf{r} - \mathbf{r}'|^3} (\mathbf{r} - \mathbf{r}')\]

We will find later that the
\href{https://en.wikipedia.org/wiki/Electric_potential}{electric
potential} (\(V\)) is a better choice if we can use it. It simplifies
our vector problem to a scalar problem.

\subsubsection{The magnetic field}\label{the-magnetic-field}

The relevant Maxwell equations for \(\mathbf{B}\) that govern an
electrostatic situation are the following:

\[\nabla \cdot \mathbf{B} = 0\]

\[\nabla \times \mathbf{B} = 0\]

Notice that these equations are solved when \(\mathbf{B}\) = 0. This is
because there are no magnetic monopoles, so if the divergence and curl
of the field are zero everywhere, then the field must be zero
everywhere. There's simply no sources of magnetic field. So in an
electrostatic situation, the magnetic field is zero and we are purely
focused on the electric field.

\subsection{Magnetostatic fields}\label{magnetostatic-fields}

In a magnetostatic situation, there have to be currents; that is, moving
charges. That is the only make to generate magnetic field from
particles. So you are going to have a non-zero current density. But we
argue that the currents are steady, so
\(\mathbf{J}(\mathbf{r},t) = \mathbf{J}(\mathbf{r})\). Now, of course,
there will be electric field, but we will discuss how we consider that
in a moment.

\subsubsection{The magnetic field}\label{the-magnetic-field-1}

The relevant Maxwell equations for \(\mathbf{B}\) that govern a
magnetostatic situation are the following:

\[\nabla \cdot \mathbf{B} = 0\]

\[\nabla \times \mathbf{B} = \mu_0 \mathbf{J}\]

The first equation always holds; there are no
\href{https://en.wikipedia.org/wiki/Magnetic_monopole}{magnetic
monopoles}, despite much research into the area
\href{https://en.wikipedia.org/wiki/Magnetic_monopole\#Searches_for_magnetic_monopoles}{including
searches} and a
\href{https://en.wikipedia.org/wiki/Magnetic_monopole\#In_SI_units}{theory
of electromagnetism} that posits them. Here's a nice article summarizing
the state of the magnetic monopole research from 2016,
\href{https://phys.org/news/2016-08-mysterious-magnetic-monopole.html}{The
mysterious missing magnetic monopole}, and an
\href{https://icecube.wisc.edu/news/research/2022/01/icecube-and-the-mystery-of-the-missing-magnetic-monopoles/}{article
from the IceCube experiment}, which a number of our
\href{https://pa.msu.edu/high-energy-physics/icecube-neutrino-observatory.aspx}{faculty,
grad students, and undergrads are involved in}.

The second equation is
\href{https://en.wikipedia.org/wiki/Amp\%C3\%A8re\%27s_circuital_law}{Ampere's
Law} in it's differential form. Much like Gauss's law, it is always
true, but rarely useful outside of specific geometries. Note that for
the magnetostatic situation that current density can vary with space
only. The integral form of Ampere's Law is can be used to solve problems
with symmetry:

\[\oint \mathbf{B} \cdot d\mathbf{l} = \mu_0 I_{enc}\]

And it is from considering this form for a single current carrying wire
that we can determine the magnetic field produced by a single current
carrying wire, \(I\), at a location \(\mathbf{r}'\):

\[\mathbf{B}(\mathbf{r}) = \frac{\mu_0 I}{4\pi} \frac{\hat{\mathbf{\phi}}}{|\mathbf{r} - \mathbf{r}'|^2}\]

Because of the curly nature of magnetic field, we find that the general
solution to the magnetostatic problem is given by the
\href{https://en.wikipedia.org/wiki/Biot\%E2\%80\%93Savart_law}{Biot-Savart
Law}:

\[d\mathbf{B} = \frac{\mu_0}{4\pi} \frac{I d\mathbf{l} \times (\mathbf{r} - \mathbf{r}')}{|\mathbf{r} - \mathbf{r}'|^3}\]

\[\mathbf{B}(\mathbf{r}) = \frac{\mu_0}{4\pi} \int \frac{I d\mathbf{l} \times (\mathbf{r} - \mathbf{r}')}{|\mathbf{r} - \mathbf{r}'|^3}\]

In practice, we rarely use this form of the qeuation directly because it
requires integrable solutions. Instead, we use the integral form of
Ampere's Law to solve problems with symmetry, we use the differential
forms to solve problems numerically, and we develop other theoretical
approaches. Because the divergence of the magnetic field is zero, we can
write the magnetic field as the curl of a
\href{https://en.wikipedia.org/wiki/Magnetic_vector_potential}{vector
potential}, \(\mathbf{B} = \nabla \times \mathbf{A}\). This has a
integral form that is similar to the Biot-Savart Law, but directly
proportional to the current density, which is much nicer:

\[\mathbf{A} = \frac{\mu_0}{4\pi} \int_V \frac{\mathbf{J}(\mathbf{r}')}{|\mathbf{r} - \mathbf{r}'|} d\tau'\]

It turns out this is a better method for solving problems, especially
computationally. That is beyond the scope of this course.

\subsubsection{The electric field}\label{the-electric-field-1}

Because there is a magnetic field, the relevant Maxwell equations for
\(\mathbf{E}\) that govern a magnetostatic situation are the following:

\[\nabla \cdot \mathbf{E} = \frac{\rho}{\varepsilon_0}\]

\[\nabla \times \mathbf{E} = -\frac{\partial \mathbf{B}}{\partial t}\]

Notice now we have to consider a changing magnetic field. In the
magnetostatic situation, we argue the current densities are setup such
that the magnetic field is static (i.e.,
\(\frac{\partial \mathbf{B}}{\partial t} = 0\)). This means the relevant
equations for the electric field remain those in the electrostatic
situation:

\[\nabla \cdot \mathbf{E} = \frac{\rho}{\varepsilon_0}\]

\[\nabla \times \mathbf{E} = 0\]

And all the relevant tools we will develop for the electrostatic
situation for \(\mathbf{E}\) apply here as well.

\subsection{Additional Resources}\label{additional-resources}

\subsubsection{Handwritten Notes}\label{handwritten-notes}

\begin{itemize}
\tightlist
\item
  \href{../../assets/notes/Notes-Vector_Calculus.pdf}{Vector Calculus}
\item
  \href{../../assets/notes/Notes-Electrostatics_and_Superposition.pdf}{Electrostatics
  and Superposition}
\item
  \href{../../assets/notes/Notes-Gauss_Law.pdf}{Gauss's Law}
\item
  \href{../../assets/notes/Notes-Magnetostatics.pdf}{Magnetostatics}
\item
  \href{../../assets/notes/Notes-Amperes_Law.pdf}{Ampere's Law}
\end{itemize}
