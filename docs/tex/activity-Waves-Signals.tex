\section{31 Oct 23 - Activity: Signal
Deconstruction}\label{oct-23---activity-signal-deconstruction}

We have started to show that we can take waves of any periodicity and
add them together to get new signals. But more common is for us to
receive a signal and want to know what waves are in it. This is the
process of signal deconstruction. We will be using \(V(t)\) to describe
this signal because pretty much everything is a voltage. But now we
introduce the toolkit that can help us do this.

We can build up this idea by starting with a given \(V(t)\) with some
base periodicity (\(w_0\) or \(T_0\)), really the longest period or
lowest frequency in the signal. We can use a Fourier Series to decompose
those signals in terms of harmonics (\(n\;\omega_0\)) of the base
frequency. The model of this decomposition is given by the
\href{https://en.wikipedia.org/wiki/Fourier_series\#Sine-cosine_form}{infinite
series}:

\(f(t) = \dfrac{a_0}{2} + \sum_{n=1}^{\infty} a_n \cos(n\omega_0t) + b_n \sin(n\omega_0t)\)

Or

\(f(t) = \dfrac{a_0}{2} + \sum_{n=1}^{\infty} a_n \cos\left(\dfrac{2n\pi}{T_0}t\right) + b_n \sin\left(\dfrac{2n\pi}{T_0}t\right)\)

By using the
\href{https://en.wikipedia.org/wiki/Orthogonal_functions}{orthogonal
nature} of \(\sin\) and \(\cos\) over one period, we can find the
unknown coefficients for a given signal \(V(t)\).

\subsection{DC Offset}\label{dc-offset}

The first term in the sum is just related to the DC offset, that is the
average of the signal over one period. That's because \(sin\) and
\(cos\) don't shift the average (they have a time average of zero,
remember?)

\(\dfrac{a_0}{2} = \dfrac{1}{T_0}\int_{T_0} f(t) dt\)

\subsection{Periodic Models}\label{periodic-models}

For the periodicity, we use ``Fourier's Trick'' to determine the value
of the coefficients. \emph{Note that you need the signal \(V(t)\) to
complete this work.} And this isn't a trick, it's a well-established
method.

\(a_n = \dfrac{2}{T_0}\int_{T_0}V(t)\cos(n\omega_0t)\qquad n\neq0\)

\(b_n = \dfrac{2}{T_0}\int_{T_0}V(t)\sin(n\omega_0t)\qquad n\neq0\)

Given this, our approximate \(V(t)\) is given up to the \(N\) term is,

\(V(t) \approx \dfrac{a_0}{2} + \sum_{n=1}^{N} a_n \cos(n\omega_0t) + b_n \sin(n\omega_0t)\)

\subsection{Example: The Duty Cycle}\label{example-the-duty-cycle}

There are many signals we could analyze, but one of the common (and more
mathematically tractable) ones is the
\href{https://en.wikipedia.org/wiki/Duty_cycle}{Duty Cycle}. In fact,
this signal is in wide use across electrical devices for timed on and
off states. Moreover, the behaviors of some neurons and muscle fibers
have been shown to approximate duty cycle firing. Synthesizers and
electronic music are similar heavy users of
\href{https://en.wikipedia.org/wiki/Pulse-width_modulation}{pulse width
modulation}, which stems from this signal.

The half-wave (square wave) duty cycle is given by,

\(V(t) = \begin{cases} 0 & 0 \leq t < \dfrac{T_0}{2} \\ V_0 & \dfrac{T_0}{2} \leq t < T_0 \end{cases}\)

\textbf{✅ Do this} 1. Identify which integrals you have to do (consider
the orthogonality of the functions). 2. Construct the integrals you need
to do (over what cycle are you integrating?). 3. Perform the integrals
(consider using \texttt{sympy}). 4. Determine the coefficients \(a_n\)'s
and \(b_n\)'s. 5. Write the function \(V(t)\) in terms of the \(a_n\)'s
and \(b_n\)'s.

\subsubsection{Visualizing the Duty
Cycle}\label{visualizing-the-duty-cycle}

It's great that we can produce a solution to this problem,

\[V(t) = \dfrac{V_0}{2} + \sum_{n=1}^{\infty} \dfrac{V_0}{n\pi}\left[\cos(n\pi)-1\right]\sin\left(\dfrac{2n\pi}{T_0}t\right)\]

but what does it look like?

\textbf{✅ Do this} 1. Write the relevant functions to compute the
coefficients \(a_n\)'s and \(b_n\)'s. 2. Write the relevant functions to
compute the \(V(t)\). 3. Plot the \(V(t)\) along with the square wave
(look into the \texttt{scipy.signal.square} function).

\begin{Shaded}
\begin{Highlighting}[]
\ImportTok{import}\NormalTok{ numpy }\ImportTok{as}\NormalTok{ np}
\ImportTok{import}\NormalTok{ matplotlib.pyplot }\ImportTok{as}\NormalTok{ plt}
\ImportTok{import}\NormalTok{ scipy.signal }\ImportTok{as}\NormalTok{ sig}
\OperatorTok{\%}\NormalTok{matplotlib inline}
\end{Highlighting}
\end{Shaded}

\begin{Shaded}
\begin{Highlighting}[]
\CommentTok{\#\# your code here}
\end{Highlighting}
\end{Shaded}

\subsection{Complex Numbers}\label{complex-numbers}

One the most useful approaches to representing periodic signals is to
use complex numbers. This is because the
\href{https://en.wikipedia.org/wiki/Euler\%27s_formula}{Euler's Formula}
allows us to represent the trigonometric functions as complex
exponentials. Euler's formula is given by,

\(e^{i\theta} = \cos(\theta) + i\sin(\theta)\)

where \(i\) is the imaginary number. But what are the properties of
\href{https://en.wikipedia.org/wiki/Complex_number}{complex numbers}?
There are a larger class of numbers than the reals, and they have
interesting properties. The generic form of a complex number is given
by,

\[z = a + ib\]

where \(a\) and \(b\) are real numbers and \(i\) is the square root of
negative one. We can also write this in terms of the magnitude and
phase,

\[z = r e^{i\theta}\]

where \(r\) is the magnitude and \(\theta\) is the phase and both are
real numbers. These two representations are compatible if we think of a
complex number represented in the complex plane where the reals appears
on the \(x\) and the imaginaries on the \(y\).

\begin{figure}
\centering
\pandocbounded{\includegraphics[keepaspectratio,alt={Complex Plane}]{https://upload.wikimedia.org/wikipedia/commons/d/d6/Argandgaussplane.png}}
\caption{Complex Plane}
\end{figure}

The basic properties of complex numbers are below for two numbers
\(z_1 = a_1 + ib_1\) and \(z_2 = a_2 + ib_2\):

\begin{itemize}
\tightlist
\item
  Addition: \(z_1 + z_2 = (a_1 + a_2) + i(b_1 + b_2)\)
\item
  Subtraction: \(z_1 - z_2 = (a_1 - a_2) + i(b_1 - b_2)\)
\item
  Multiplication:
  \(z_1 z_2 = (a_1 a_2 - b_1 b_2) + i(a_1 b_2 + a_2 b_1)\)
\item
  Division:
  \(\dfrac{z_1}{z_2} = \dfrac{a_1 a_2 + b_1 b_2}{a_2^2 + b_2^2} + i\dfrac{a_2 b_1 - a_1 b_2}{a_2^2 + b_2^2}\)
\end{itemize}

Some of these operations are simpler in the polar form. For example,
multiplication is just the product of the magnitudes and the sum of the
phases. Division is the ratio of the magnitudes and the difference of
the phases. For two complex numbers \(z_1 = r_1 e^{i\theta_1}\) and
\(z_2 = r_2 e^{i\theta_2}\):

\begin{itemize}
\tightlist
\item
  Multiplication: \(z_1 z_2 = r_1 r_2 e^{i(\theta_1 + \theta_2)}\)
\item
  Division:
  \(\dfrac{z_1}{z_2} = \dfrac{r_1}{r_2} e^{i(\theta_1 - \theta_2)}\)
\end{itemize}

Other operations are best represented in the polar form:

\begin{itemize}
\tightlist
\item
  Powers: \(z^n = r^n e^{in\theta}\)
\item
  Roots: \(z^{1/n} = r^{1/n} e^{i\theta/n}\)
\end{itemize}

\subsubsection{Representing Trigonometric
Functions}\label{representing-trigonometric-functions}

We can use Euler's formula to represent the trigonometric functions as
complex exponentials. For example,

\[\cos(\theta) = \dfrac{e^{i\theta} + e^{-i\theta}}{2}\]
\[\sin(\theta) = \dfrac{e^{i\theta} - e^{-i\theta}}{2i}\]

Take the square wave function from above:

\[V(t) = \begin{cases} 0 & 0 \leq t < \dfrac{T_0}{2} \\ V_0 & \dfrac{T_0}{2} \leq t < T_0 \end{cases}\]

and compute the complex coefficients \(c_n\)'s,

\[c_n = \dfrac{1}{T_0}\int_{T_0}V(t)e^{-in\omega_0t}dt\]

where our new model is,

\[V(t) = a_0 + \sum_{n=-\infty}^{\infty} c_n e^{in\omega_0t}\]

\textbf{✅ Do this}

\begin{enumerate}
\def\labelenumi{\arabic{enumi}.}
\tightlist
\item
  Take the signal \(V(t)\) and determine which integrals you have to do.
\item
  Construct the integrals you need to do (over what cycle are you
  integrating?).
\item
  Perform the integrals (consider using \texttt{sympy}) and find the
  coefficients \(c_n\)'s.
\item
  Write the function \(V(t)\) in terms of the \(c_n\)'s. (What about the
  imaginary part?)
\end{enumerate}
