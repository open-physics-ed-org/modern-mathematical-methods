\section{7 Sept 23 - Notes: Numerical Integration in
1D}\label{sept-23---notes-numerical-integration-in-1d}

Principal Ideas for this unit will come from
\href{https://github.com/dannycab/phy415msu/blob/main/MMIPbook/assets/pdfs/scans/Newman_Ch8_ODEs.pdf}{Newman
Chapter 8, Section 1}, which is the reading assigned below.

We have seen how we can quickly exhaust our analytical tools when we
investigate systems with interesting and complex behavior. We can see
much from the perspective of phase space in terms of the qualitatively
different behaviors (e.g., oscillations vs rotations, both clockwise and
counter-clockwise, in the large angle pendulum). But what if we want to
find the specific trajectory of a given set of initial conditions.

That is we want to be able to take a more general differential equation:

\[\ddot{x} = f(x,\dot{x},t)\]

to a solution for all (or some specified amount)of time,
\(x(t)\)\ldots in (most) cases\ldots because sometimes integrals don't
converge. We can use \textbf{numerical integration} to find these
trajectories.

\subsection{Numerical Integration}\label{numerical-integration}

\href{https://en.wikipedia.org/wiki/Numerical_integration}{Numerical
Integration} is a vast and wide topic with lots of different approaches,
important nuances, and difficult problems. Some of the most high profile
numerical integration was done by NASA's
\href{https://education.nationalgeographic.org/resource/women-nasa}{human
computers} -- a now well-known story thanks to the film
\href{https://en.wikipedia.org/wiki/Hidden_Figures}{Hidden Figures}.
Black women formed a core group of these especially talented scientists
(including
\href{https://en.wikipedia.org/wiki/Mary_Jackson_(engineer)}{Mary
Jackson},
\href{https://en.wikipedia.org/wiki/Katherine_Johnson}{Katherine
Johnson}, and
\href{https://en.wikipedia.org/wiki/Dorothy_Vaughan}{Dorothy Vaughn}),
without whom, John Glenn would not have orbited the Earth in 1962. This
is also a very interesting story about the importance of
\href{https://en.wikipedia.org/wiki/Historically_black_colleges_and_universities}{Historically
Black Colleges and Universities} to American science.

Below is a video describing 3 coupled ODEs, the
\href{https://en.wikipedia.org/wiki/Lorenz_system}{Lorenz equations},
that are quite famous. You will be able to model systems like this, but
for now think of it as motivation for why we want to learn numerical
integration.

\href{https://inv.tux.pizza/watch?v=aAJkLh76QnM}{\pandocbounded{\includegraphics[keepaspectratio]{https://markdown-videos-api.jorgenkh.no/youtube/aAJkLh76QnM?width=720&height=405}})}
- Non-Commercial Link: \url{https://inv.tux.pizza/watch?v=aAJkLh76QnM?}
- Commercial Link: \url{https://youtube.com/watch?v=aAJkLh76QnM?}

\subsection{Nonlinear Science}\label{nonlinear-science}

Some of the most interesting ODEs to work with are those that lead to
chaos, have interesting behaviors in different parts of phase space, are
highly sensitive to initial conditions, or that have complex bound
orbits (in real or phase space). This area of study in physics overlaps
with a broad interdisciplinary field called
\href{https://en.wikipedia.org/wiki/Nonlinear_system}{Nonlinear
Science}, which studies all manner of systems where the results are not
simply proportional to the input features. This work includes theory,
applied science, and experimental work. As it is difficult to find truly
linear systems in the real world, nonlinear science is quite
interdisciplinary with physics research in the areas of
\href{https://complex.umd.edu/research/MHD_dynamos/MHD_dynamos.php}{magnetohydrodynamics},
\href{}{fluid mechanics},
\href{https://meche.mit.edu/people/faculty/PEKO@MIT.EDU}{bio-inspired
design}, \href{https://curtisresearch.gatech.edu/index.php}{biophysics
(cellular)}, \href{https://crablab.gatech.edu/}{biophysics (animal
locomotion)}, \href{https://jila.colorado.edu/lewandowski}{atomic and
molecular physics}, and many more fields.

We will focus on integrating 1 dimensional ODEs (that is, ODEs of only
one variable like the SHO). To that end the reding this week aims to
inform you with the simplest of the integrators the
\href{https://en.wikipedia.org/wiki/Semi-implicit_Euler_method}{Euler-Cromer
integration} technique. This approach is intuitive, straight-forward,
and forms the basis for better and far more accurate methods like
\href{https://en.wikipedia.org/wiki/Runge\%E2\%80\%93Kutta_methods}{Runge-Kutta}.

Through this week's activities, we will introduce several integration
methods, but you will learn to use the
\href{https://docs.scipy.org/doc/scipy/reference/generated/scipy.integrate.odeint.html?highlight=odeint}{built-in
integrator} (\texttt{odeint}) from scipy, which are more efficient than
what we could write ourselves.

\subsection{How Does Numerical Integration actually
work?}\label{how-does-numerical-integration-actually-work}

A computer understands things like updating individual variables with a
change. It turns out this process of updating things in steps is the
basis for numerical integration. We need a set of update equations.
Making those update equations is effectively choosing our integrator.

\subsubsection{Update equations}\label{update-equations}

The critical part of numerical integration is approximating the change
to variables you are investigating. Going back to our differential
equations, we can rewrite them as approximate equation, which a computer
understands because it involves discrete steps. How we choose to
approximate this update indicates which integration routine we've chosen
and sets the irreducible error we are stuck with (i.e.,
\(O((\Delta t)^2)\), \(O((\Delta t)^3)\), etc.).

My handwritten notes (below) provide derivations for the Euler-Cromer
method. Other forms of integration are more complex, but the basic idea
is the same: \emph{we are trying to approximate the ``area under the
curve'' by sampling the slope of the function at different points.}

We will illustrate three approximations to the slope of these functions:

\begin{itemize}
\tightlist
\item
  \textbf{Euler-Cromer (EC)} - definitely the most intuitive of the
  approaches, where we approximate the slope with two points separated
  by \(\Delta t\) in time. It is quick to write, slow to solve, and
  requires small steps for accurate results. Even so, it fails to
  integrate periodic motion well because it doesn't always conserve
  energy in periodic motion. Turns out it's the best tool to use when
  you have random noise added to the model though (e.g.,
  \(\eta_n(\sigma(t))\)). For a first order eqn, \(\dot{x}=f(x,t)\),
\end{itemize}

\[x(t+\Delta t) = x(t) + \textrm{change} = x(t) + \Delta t \left(f(x(t+\dfrac{1}{2}\Delta t), t+\dfrac{1}{2}\Delta t\right)\]

\begin{itemize}
\tightlist
\item
  \textbf{Runge-Kutta 2nd order (RK2)} - just a step above Euler-Cromer;
  it uses three points to approximate the slope giving two measures of
  the slope (hence, 2nd order). It's not much more complex than
  Euler-Cromer, but gives an order of magnitude lower error. It's a good
  starting point for simple systems. For a first order eqn,
  \(\dot{x}=f(x,t)\),
\end{itemize}

\[k_1 = \Delta t\left(f(x,t)\right),\]
\[k_2 =  \Delta t\left(x+\dfrac{1}{2}k_1, t+\dfrac{1}{2}\Delta t\right),\]
\[x(t+\Delta t) = x(t) + \textrm{change} = x(t) + k_2\]

\begin{itemize}
\tightlist
\item
  \textbf{Runge Kutta 4th order (RK4)} - this is the gold standard. Most
  researchers start with RK4 on most problems. It uses 5 points to build
  4 slope profiles and integrates the system in 4 steps. It is highly
  adaptable and supported -- it can be modified to take smaller or
  longer steps depending on the specific nature of the problem at the
  time. I mean that it can change step size in the middle of its work;
  including within the step it is taking presently. For a first order
  eqn, \(\dot{x}=f(x,t)\),
\end{itemize}

\[k_1 = \Delta t\left(f(x,t)\right),\]
\[k_2 =  \Delta t\left(x+\dfrac{1}{2}k_1, t+\dfrac{1}{2}\Delta t\right),\]
\[k_3 =  \Delta t\left(x+\dfrac{1}{2}k_2, t+\dfrac{1}{2}\Delta t\right),\]
\[k_4 =  \Delta t\left(x+k_3, t+\Delta t\right),\]
\[x(t+\Delta t) = x(t) + \textrm{change} = x(t) + \dfrac{1}{6}\left(k_1 + 2k_2 +2k_3 +k_4\right)\]

\textbf{We don't expect you memorize these approaches or to derive them,
but to understand how they work and what their limitations are.} We will
implement them in class using known tools.

\subsection{Additional Resources}\label{additional-resources}

\subsubsection{Handwritten Notes}\label{handwritten-notes}

\begin{itemize}
\tightlist
\item
  \href{../../assets/notes/Notes-Numerical_Integration_1D_Spatial.pdf}{Numerical
  Integration - 1D Spatial integrals}
\item
  \href{../../assets/notes/Notes-Numerical_Integration_Euler_Cromer.pdf}{Numerical
  Integration - Euler Cromer for ODEs}
\end{itemize}

\subsubsection{Book Readings}\label{book-readings}

These two readings are useful preparation for what we are going to do
with numerical integration.

\subsubsection{Newman's Computational Physics - Chapter
8}\label{newmans-computational-physics---chapter-8}

The first one is from
\href{https://github.com/dannycab/phy415msu/blob/main/MMIPbook/assets/pdfs/scans/Newman_Ch8_ODEs.pdf}{Chapter
8 of Mark Newman's book, Computational Physics}, which is an excellent
introduction to most of the computational physics approaches we use in
physics. It uses Python and modern libraries, which is rare for a
computational physics textbook for undergraduate students.

\textbf{Read This} Sections 8.1 and 8.1.1. We will discuss and work with
ideas from 8.1.2 and 8.1.3, but I don't expect you to read those in
detail. We will discuss how these work and then make use of the
\href{https://docs.scipy.org/doc/scipy/tutorial/integrate.html}{built-in
integrators from scipy}. The core one will be \texttt{odeint}
\href{https://docs.scipy.org/doc/scipy/reference/generated/scipy.integrate.odeint.html}{(Documentation)}.

\subsection{Paper Readings}\label{paper-readings}

Here's the original paper from Alan Cromer about the correction to the
Euler step that makes it more accurate. It's a good read, but not
required.

\begin{itemize}
\tightlist
\item
  \href{https://github.com/dannycab/phy415fall23/blob/main/content/assets/papers/euler_cromer_1981.pdf}{Cromer,
  Alan H. ``Stable solutions using the Euler approximation.'' American
  Journal of Physics 49.5 (1981): 455-459.}
\end{itemize}
