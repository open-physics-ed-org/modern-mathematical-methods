\section{6 Oct 23 - Notes: Fast Fourier
Transform}\label{oct-23---notes-fast-fourier-transform}

It might not have been obvious to you, but our work so far has expected
that our \(V(t)\) be a continuous function of time, and that we can
sample it any time we want. In reality, you are limited to the response
of your equipment, which leads to a phenomenon knows as the
\href{https://en.wikipedia.org/wiki/Nyquist\%E2\%80\%93Shannon_sampling_theorem}{Nyquist-Shannon
sampling theorem}. This theorem states that if you want to study a
signal with a frequency \(f\), you need to sample it at least \(2f\)
times per second. This is known as the
\href{https://en.wikipedia.org/wiki/Nyquist_frequency}{Nyquist
frequency}. If you don't sample at least this often, you will get
aliasing, which is when a signal at a higher frequency is
indistinguishable from a signal at a lower frequency. As you can tell,
there's a lot of issues with sampling when it comes to studying signals.

\subsection{The Fast Fourier
Transform}\label{the-fast-fourier-transform}

This video from Veritasium really explains the importance of this
algorithm. It's a great watch.

\href{https://inv.tux.pizza/watch?v=nmgFG7PUHfo}{\pandocbounded{\includegraphics[keepaspectratio,alt={History of the FFT}]{https://markdown-videos-api.jorgenkh.no/youtube/nmgFG7PUHfo?width=720&height=405}}}

\begin{itemize}
\tightlist
\item
  Non-Commercial Link: \url{https://inv.tux.pizza/watch?v=nmgFG7PUHfo}
\item
  Commercial Link: \url{https://youtube.com/watch?v=nmgFG7PUHfo}
\end{itemize}

\subsection{The Discrete Fourier
Transform}\label{the-discrete-fourier-transform}

The foundation of this algorithm is the
\href{https://en.wikipedia.org/wiki/Discrete_Fourier_transform}{Discrete
Fourier Transform}. This is a mathematical operation that takes a
discrete signal and transforms it into a continuous function of
frequency. The equation for this is:

We start with complex form of the approximation of the Fourier Series:

\[f(t) = \sum_{n=-\infty}^{\infty} c_n e^{i n \omega_0 t}\]

Where \(c_n\) is the complex coefficient of the \(n\)th harmonic, and
\(\omega_0\) is the fundamental frequency. As we have seen before, we
can rewrite this as:

\[c_n = \frac{1}{T_0} \int_{0}^{0 + T_0} f(t) e^{-i n \omega_0 t} dt\]

Where \(T_0\) is the period of the signal. Sometimes that integral is
analytical and sometimes it isn't. If it's real data, it is definitely
questionable if it's analytical. So, we can approximate this integral
using the trapezoidal rule:

\[c_n \approx \frac{1}{T_0} \sum_{k=0}^{N-1} f(t_k) e^{-i n \omega_0 t_k} \Delta t\]

Where \(N\) is the number of samples, \(t_k\) is the time of the \(k\)th
sample, and \(\Delta t\) is the time between samples. We then use this
to compute the approximate signal:

\[f(t) \approx \sum_{n=-\infty}^{\infty} \left( \frac{1}{T_0} \sum_{k=0}^{N-1} f(t_k) e^{-i n \omega_0 t_k} \Delta t \right) e^{i n \omega_0 t}\]

It's great that the numerical technique is well known and canned tools
are available like \texttt{scipy.integrate.trapz}, but here we will
start to use \texttt{scipy.fft} to do the heavy lifting for us.

\subsection{Additional Resources}\label{additional-resources}

We do not expect you to understand the details of an FFT, how it is
derived and how it works when written into a computer program. However,
if you are interested, here are some resources that you can use to learn
more about the FFT.

\subsubsection{Videos}\label{videos}

This is an excellent discussion of the underlying mathematics and the
setup of the algorithm.

\href{https://inv.tux.pizza/watch?v=h7apO7q16V0}{\pandocbounded{\includegraphics[keepaspectratio,alt={Mathematics of the FFT}]{https://markdown-videos-api.jorgenkh.no/youtube/h7apO7q16V0?width=720&height=405}}}

\begin{itemize}
\tightlist
\item
  Non-Commercial Link: \url{https://inv.tux.pizza/watch?v=h7apO7q16V0}
\item
  Commercial Link: \url{https://youtube.com/watch?v=h7apO7q16V0}
\end{itemize}

This is a presentation of how to implement the FFT using Python code.

\href{https://inv.tux.pizza/watch?v=Ty0JcR6Dvis}{\pandocbounded{\includegraphics[keepaspectratio,alt={Implementing FFT}]{https://markdown-videos-api.jorgenkh.no/youtube/Ty0JcR6Dvis?width=720&height=405}}}

\begin{itemize}
\tightlist
\item
  Non-Commercial Link: \url{https://inv.tux.pizza/watch?v=Ty0JcR6Dvis}
\item
  Commercial Link: \url{https://youtube.com/watch?v=Ty0JcR6Dvis}
\end{itemize}

\subsubsection{Handwritten notes}\label{handwritten-notes}

\begin{itemize}
\tightlist
\item
  \href{../assets/notes/Notes-Introduction_to_FFT.pdf}{Introduction to
  FFT}
\end{itemize}
