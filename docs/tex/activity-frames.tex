\section{31 Aug 23 - Activity: Frames and
Coordinates}\label{aug-23---activity-frames-and-coordinates}

\subsection{Forces in different
frames}\label{forces-in-different-frames}

One of the critical things that we must note is that the natural world
changes and we observe it. The models (or laws) that we use to describe
nature must accurately describe our observations. This is challenging
when observers use different frames (or even different coordinates in
the same frame) to show we have the same observations. Moving between
coordinate systems and frames is a critical skill in physics.

\subsection{Polar coordinates}\label{polar-coordinates}

Many problems in physics require the use of non-Cartesian coordinates,
such as the Hydrogen atom or the two-body problem. One such coordinate
system is \textbf{plane-polar coordinates}. In this coordinate system,
any vector \(\mathbf{r}\in \mathbb{R}^2\) is described by a distance
\(r\) and angle \(\phi\) instead of Cartesian coordinates \(x\) and
\(y\). The following four equations show how points transform in these
coordinate systems.

\[
x = r\cos \phi \hspace{1in} y = r\sin \phi
\]

\[
r = \sqrt{x^2 + y^2} \hspace{1in} \phi = \arctan(y / x)
\]

\subsubsection{Getting Oriented}\label{getting-oriented}

\textbf{✅ Do this}

Borrowed from
\href{https://physicscourses.colorado.edu/EducationIssues/ClassicalMechanics/}{CU
Boulder Physics}

A particle moves in the plane. We could describe its motion in two
different ways:

\textbf{CARTESIAN}: I tell you \(x(t)\) and \(y(t)\).

\textbf{POLAR}: I tell you \(r(t)\) and \(\phi(t)\). (Here \(r(t)\) =
\(|\mathbf{r}(t)|\), it's the ``distance to the origin'')

\begin{itemize}
\tightlist
\item
  \begin{enumerate}
  \def\labelenumi{(\alph{enumi})}
  \tightlist
  \item
    Draw a picture showing the location of the point at some arbitrary
    time, labeling \(x, y, r, \phi\) and also showing the unit vectors
    \(\hat{x}, \hat{y}, \hat{r},\) and \(\hat{\phi}\), all at this one
    time.
  \end{enumerate}
\item
  \begin{enumerate}
  \def\labelenumi{(\alph{enumi})}
  \setcounter{enumi}{1}
  \tightlist
  \item
    Using this picture, determine the formula for \(\hat{r}(t)\) in
    terms of the Cartesian unit vectors. You answer should contain
    \(\phi(t)\).
  \end{enumerate}
\item
  \begin{enumerate}
  \def\labelenumi{(\alph{enumi})}
  \setcounter{enumi}{2}
  \tightlist
  \item
    Write down the analogous expression for \(\hat{\phi}(t)\).
  \end{enumerate}
\item
  \begin{enumerate}
  \def\labelenumi{(\alph{enumi})}
  \setcounter{enumi}{3}
  \tightlist
  \item
    We can claim the position vector in Cartesian coordinates is
    \(\vec{r}(t) = x(t)\hat{x} + y(t)\hat{y}\). Do you agree? Is this
    consistent with your picture above?
  \end{enumerate}
\item
  \begin{enumerate}
  \def\labelenumi{(\alph{enumi})}
  \setcounter{enumi}{4}
  \tightlist
  \item
    We can claim the position vector in polar coordinates is just
    \(\vec{r}(t) = r(t)\hat{r}\). Again, do you agree? Why isn't there a
    \(+\phi(t)\hat{\phi}\) term?
  \end{enumerate}
\end{itemize}

\subsubsection{Getting Kinetic}\label{getting-kinetic}

\textbf{✅ Do this}

\begin{itemize}
\tightlist
\item
  \begin{enumerate}
  \def\labelenumi{(\alph{enumi})}
  \tightlist
  \item
    Now let's find the velocity, \(\vec{v}(t) = d\vec{r}/dt\). In
    Cartesian coordinates, it's just
    \(\vec{v}(t) = \dot{x}(t)\hat{x} + \dot{y}(t)\hat{y}\). Explain why,
    in polar coordinates, the velocity can be written as
    \(d\vec{r}/dt = r(t)\:d\hat{r}/dt + dr(t)/dt\:\hat{r}\).
  \end{enumerate}
\item
  \begin{enumerate}
  \def\labelenumi{(\alph{enumi})}
  \setcounter{enumi}{1}
  \tightlist
  \item
    It appears we need to figure out what \(d\hat{r}/dt\) is. Use the
    formula your determined in question 1b to get started -- first in
    terms of \(\hat{x}\) and \(\hat{y}\), and then converting to pure
    polar.
  \end{enumerate}
\item
  \begin{enumerate}
  \def\labelenumi{(\alph{enumi})}
  \setcounter{enumi}{2}
  \tightlist
  \item
    Write an expression for \(\vec{v}(t)\) in polar coordinates.
  \end{enumerate}
\item
  \begin{enumerate}
  \def\labelenumi{(\alph{enumi})}
  \setcounter{enumi}{3}
  \tightlist
  \item
    Finally, determine the acceleration \(\vec{a} = d\vec{v}(t)/dt\). In
    Cartesian coordinates, it's just
    \(\vec{a}(t) = \ddot{x(t)}\hat{x} + \ddot{y}(t)\hat{y}\). Work it on
    in polar coordinates.
  \end{enumerate}
\end{itemize}

\subsection{Forces and acceleration in plane-polar
coordinates}\label{forces-and-acceleration-in-plane-polar-coordinates}

We can show that the acceleration in plane-polar coordinates is given
by:

\[\mathbf{a} = a_r\hat{r} + a_{\phi}\hat{\phi} = \left(\ddot{r}-r\dot{\phi}^2\right)\hat{r} + \left(r\ddot{\phi}+2\dot{r}\dot{\phi}\right)\hat{\phi}\]

Because this coordinate system is orthgonal
(\(\hat{r}\cdot\hat{\phi} = 0\)), we can write the Newton's second law
in this coordinate system as:

\[\mathbf{F}_{net} = m\mathbf{a} = m\left(a_r\hat{r} + a_{\phi}\hat{\phi} \right)\]

So that,

\[\mathbf{F}_r = m\left(\ddot{r}-r\dot{\phi}^2\right)\hat{r}\]

and

\[\mathbf{F}_{\phi} = m\left(r\ddot{\phi}+2\dot{r}\dot{\phi}\right)\hat{\phi}\]

\subsubsection{Example to Work in a
Group}\label{example-to-work-in-a-group}

\textbf{✅ Do this}

Borrowed from
\href{https://bookshop.org/p/books/classical-mechanics-john-r-taylor/17213436?ean=9781891389221}{Taylor's
Classical Mechanics}

Consider a ``half-pipe'' that has a circular cross section of radius
\(R\). If we release the skateboard near the bottom of the ``half-pipe''
approximately how long does it take to get to the bottom?

\begin{itemize}
\tightlist
\item
  Make sure you draw a picture and define your coordinate system.
\item
  Consider all the assumptions you need to make to solve this problem,
  and write them down.
\item
  Be careful not to make big assumptions too early; try instead to write
  the forces acting on the skateboard in plane-polar coordinates.
\item
  Do not use Lagrangian dynamics to solve this problem if you know how
  you, please instead use Newton's Second Law.
\end{itemize}

\textbf{Hint: The equation of motion to small angle oscillation
frequency pipeline is real.}

\begin{itemize}
\tightlist
\item
  \href{../../assets/notes/Notes-Newton_2nd_Plane_polar.pdf}{Notes on
  Newton's Second Law in Plane Polar Coordinates}
\end{itemize}
