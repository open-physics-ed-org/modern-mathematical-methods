\section{19 Sep 23 - The Duffing
Oscillator}\label{sep-23---the-duffing-oscillator}

We have developed a set of tools to investigate differential equations.
You will now apply those skills to the
\href{https://en.wikipedia.org/wiki/Duffing_equation}{Duffing
Oscillator}. The Duffing Oscillator is a second order differential
equation that is used to model a forced nonlinear spring. The equation
is given by:

\[\frac{d^2x}{dt^2} + \delta \frac{dx}{dt} + \alpha x + \beta x^3 = \gamma \cos(\omega t)\]

And written in the form \(\ddot{x} = f(x, \dot{x}, t)\), it is:

\[\ddot{x} = - \delta \dot{x} - \alpha x - \beta x^3 + \gamma \cos(\omega t)\]

where \(x\) is the position of the oscillator, \(\delta\) is a damping
term, \(\alpha\) is the stiffness of the spring, \(\beta\) is the
strength of the non-linear term, \(\gamma\) is the amplitude of the
driving force, and \(\omega\) is the frequency of the driving force.

\begin{figure}
\centering
\pandocbounded{\includegraphics[keepaspectratio,alt={Duffing Oscillator}]{https://upload.wikimedia.org/wikipedia/commons/f/fc/Duffing_oscillator_strange_attractor_with_color.gif}}
\caption{Duffing Oscillator}
\end{figure}

\emph{From
\href{https://commons.wikimedia.org/wiki/File:Duffing_oscillator_strange_attractor_with_color.gif}{Wikipedia}}

\subsection{Activity}\label{activity}

This is a complicated oscillator with damping, forcing, and
non-linearity. We will investigate the behavior of the oscillator for
different values of the parameters, and you will need to bring all the
tools we have used so far to bear on this problem. We will perform this
work systematically, as way of helping you recognize the steps you can
take to investigate any differential equation. \emph{You are welcome to
use any of the code that we developed or presented in class.}

\subsubsection{Organizing your analysis}\label{organizing-your-analysis}

The Duffing Oscillator has a lot of different elements and can become
complicated very quickly. To help start our analysis let's consider
three models with increasing complexity:

\begin{enumerate}
\def\labelenumi{\arabic{enumi}.}
\tightlist
\item
  \textbf{Model 1} (no damping and no forcing)
  \(\ddot{x} = - \alpha x - \beta x^3\)
\item
  \textbf{Model 2} (no forcing)
  \(\ddot{x} = - \delta \dot{x} - \alpha x - \beta x^3\)
\item
  \textbf{Model 3} (full model)
  \(\ddot{x} = - \delta \dot{x} - \alpha x - \beta x^3 + \gamma \cos(\omega t)\)
\end{enumerate}

\textbf{✅ Do this}

\subsubsection{Tasks to complete}\label{tasks-to-complete}

For each of these models, you should produce the following:

\begin{enumerate}
\def\labelenumi{\arabic{enumi}.}
\tightlist
\item
  The fixed points (if they can be found and their linear stability)
\item
  A phase portrait of the system
\item
  A graph of x(t) for a given set of initial conditions
\item
  A graph of a trajectory in the phase portrait
\item
  An approximate solution to the differential equation for a limiting
  case
\end{enumerate}

\subsubsection{Exploring your models}\label{exploring-your-models}

As you are developing your models, you should consider the following
questions:

\begin{enumerate}
\def\labelenumi{\arabic{enumi}.}
\tightlist
\item
  What kinds of motion seems possible for each model?
\item
  What are the limiting cases for each model? What solutions do they
  represent? Think about the physical meaning of the parameters.
\item
  What relationships exist between the parameters and the motion of the
  system? That is, what happens when you change the parameters?
\end{enumerate}

\begin{Shaded}
\begin{Highlighting}[]
\CommentTok{\#\# Imports to get started}

\ImportTok{import}\NormalTok{ matplotlib.pyplot }\ImportTok{as}\NormalTok{ plt}
\ImportTok{import}\NormalTok{ numpy }\ImportTok{as}\NormalTok{ np}
\ImportTok{from}\NormalTok{ scipy.integrate }\ImportTok{import}\NormalTok{ solve\_ivp}
\end{Highlighting}
\end{Shaded}

\begin{Shaded}
\begin{Highlighting}[]
\CommentTok{\#\# YOUR CODE HERE}
\end{Highlighting}
\end{Shaded}

\subsection{Routes to Chaos}\label{routes-to-chaos}

One important route to chaotic behavior is through
\href{https://en.wikipedia.org/wiki/Period-doubling_bifurcation}{period
doubling}. It is not the only way for a system to become chaotic, but it
is a characteristic way that a system will tend towards chaos.

\textbf{✅ Do this}

If your code is fully working, you can use known values of the
parameters to investigate this behavior. For this part of the activity,
you should use the full model and the following parameters:

\begin{longtable}[]{@{}ll@{}}
\toprule\noalign{}
Parameter & Value \\
\midrule\noalign{}
\endhead
\bottomrule\noalign{}
\endlastfoot
\(\alpha\) & -1 \\
\(\beta\) & +1 \\
\(\delta\) & +0.3 \\
\(\omega\) & +1.2 \\
\(\gamma\) & +0.20 to +0.65 \\
\end{longtable}

Here \(\gamma\) is the only parameter that we will change. Produce phase
plots of \(x(t)\) and \(x(\dot{x})\) for several choices of \(\gamma\)
with \(x(0) = 1\) and \(\dot{x}(0) = 0\).
