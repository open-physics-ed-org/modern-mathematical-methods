\section{7 Dec 23 - Activity: The Ising
Model}\label{dec-23---activity-the-ising-model}

\[E = -J\left(\vec{S}_i\cdot\vec{S}_j\right)\]

\begin{Shaded}
\begin{Highlighting}[]
\ImportTok{import}\NormalTok{ numpy }\ImportTok{as}\NormalTok{ np}
\ImportTok{import}\NormalTok{ matplotlib.pyplot }\ImportTok{as}\NormalTok{ plt}
\ImportTok{import}\NormalTok{ random }\ImportTok{as}\NormalTok{ random}
\end{Highlighting}
\end{Shaded}

\begin{verbatim}
---------------------------------------------------------------------------

ModuleNotFoundError                       Traceback (most recent call last)

/Users/caballero/repos/teaching/phy415fall23/content/4_distributions/activity-ising_model.ipynb Cell 2 line 1
----> <a href='vscode-notebook-cell:/Users/caballero/repos/teaching/phy415fall23/content/4_distributions/activity-ising_model.ipynb#W1sZmlsZQ%3D%3D?line=0'>1</a> import numpy as np
      <a href='vscode-notebook-cell:/Users/caballero/repos/teaching/phy415fall23/content/4_distributions/activity-ising_model.ipynb#W1sZmlsZQ%3D%3D?line=1'>2</a> import matplotlib.pyplot as plt
      <a href='vscode-notebook-cell:/Users/caballero/repos/teaching/phy415fall23/content/4_distributions/activity-ising_model.ipynb#W1sZmlsZQ%3D%3D?line=2'>3</a> import random as random


ModuleNotFoundError: No module named 'numpy'
\end{verbatim}

\begin{Shaded}
\begin{Highlighting}[]
\NormalTok{cellLength }\OperatorTok{=} \DecValTok{20}
\NormalTok{simulationSteps }\OperatorTok{=} \DecValTok{1000000}
\NormalTok{couplingConstant }\OperatorTok{=} \FloatTok{1.0} \CommentTok{\#\# J}
\NormalTok{temperature }\OperatorTok{=} \FloatTok{1.0}

\KeywordTok{def}\NormalTok{ calculateEnergy(spinArray):}
    \CommentTok{\textquotesingle{}\textquotesingle{}\textquotesingle{}Calculate all the pairwise energy interactions and sum them up}
\CommentTok{    Do rows and columns separately and add them up.\textquotesingle{}\textquotesingle{}\textquotesingle{}}
    
\NormalTok{    rowNeighborInteractionEnergy }\OperatorTok{=}\NormalTok{ np.}\BuiltInTok{sum}\NormalTok{(spinArray[}\DecValTok{0}\NormalTok{:cellLength}\OperatorTok{{-}}\DecValTok{1}\NormalTok{,:]}\OperatorTok{*}\NormalTok{spinArray[}\DecValTok{1}\NormalTok{:cellLength,:])}
\NormalTok{    columnNeighborInteractionEnergy }\OperatorTok{=}\NormalTok{ np.}\BuiltInTok{sum}\NormalTok{(spinArray[:,}\DecValTok{0}\NormalTok{:cellLength}\OperatorTok{{-}}\DecValTok{1}\NormalTok{]}\OperatorTok{*}\NormalTok{spinArray[:,}\DecValTok{1}\NormalTok{:cellLength])}
    
\NormalTok{    totalInteractionEnergy }\OperatorTok{=}\NormalTok{ rowNeighborInteractionEnergy}\OperatorTok{+}\NormalTok{columnNeighborInteractionEnergy}
    
    \ControlFlowTok{return} \OperatorTok{{-}}\NormalTok{couplingConstant}\OperatorTok{*}\NormalTok{totalInteractionEnergy}

\CommentTok{\#\# Create an empty square array}
\NormalTok{spinArray }\OperatorTok{=}\NormalTok{ np.empty([cellLength,cellLength], }\BuiltInTok{int}\NormalTok{)}

\CommentTok{\#\# Populate it with random spins}
\ControlFlowTok{for}\NormalTok{ row }\KeywordTok{in} \BuiltInTok{range}\NormalTok{(cellLength):}
    \ControlFlowTok{for}\NormalTok{ column }\KeywordTok{in} \BuiltInTok{range}\NormalTok{(cellLength):}
        \ControlFlowTok{if}\NormalTok{ random.random()}\OperatorTok{\textless{}}\FloatTok{0.5}\NormalTok{:}
\NormalTok{            spinArray[row,column] }\OperatorTok{=} \OperatorTok{+}\DecValTok{1}
        \ControlFlowTok{else}\NormalTok{:}
\NormalTok{            spinArray[row,column] }\OperatorTok{=} \OperatorTok{{-}}\DecValTok{1}

\CommentTok{\# Calculate the initial energy and magnetization        }
\NormalTok{energyAtStep }\OperatorTok{=}\NormalTok{ calculateEnergy(spinArray)}
\NormalTok{magnetizationAtStep }\OperatorTok{=}\NormalTok{ np.}\BuiltInTok{sum}\NormalTok{(spinArray)}

\CommentTok{\#\# Show the spin array }
\CommentTok{\#\# Black is spin up and white is spin down}
\NormalTok{plt.figure(figsize}\OperatorTok{=}\NormalTok{(}\DecValTok{8}\NormalTok{,}\DecValTok{8}\NormalTok{))}
\NormalTok{c }\OperatorTok{=}\NormalTok{ plt.pcolor(spinArray, cmap}\OperatorTok{=}\StringTok{\textquotesingle{}Greys\textquotesingle{}}\NormalTok{)}
\NormalTok{plt.axis(}\StringTok{\textquotesingle{}square\textquotesingle{}}\NormalTok{)}
\end{Highlighting}
\end{Shaded}

\begin{verbatim}
---------------------------------------------------------------------------

NameError                                 Traceback (most recent call last)

/Users/caballero/repos/teaching/phy415fall23/content/4_distributions/activity-ising_model.ipynb Cell 3 line 1
     <a href='vscode-notebook-cell:/Users/caballero/repos/teaching/phy415fall23/content/4_distributions/activity-ising_model.ipynb#W2sZmlsZQ%3D%3D?line=14'>15</a>     return -couplingConstant*totalInteractionEnergy
     <a href='vscode-notebook-cell:/Users/caballero/repos/teaching/phy415fall23/content/4_distributions/activity-ising_model.ipynb#W2sZmlsZQ%3D%3D?line=16'>17</a> ## Create an empty square array
---> <a href='vscode-notebook-cell:/Users/caballero/repos/teaching/phy415fall23/content/4_distributions/activity-ising_model.ipynb#W2sZmlsZQ%3D%3D?line=17'>18</a> spinArray = np.empty([cellLength,cellLength], int)
     <a href='vscode-notebook-cell:/Users/caballero/repos/teaching/phy415fall23/content/4_distributions/activity-ising_model.ipynb#W2sZmlsZQ%3D%3D?line=19'>20</a> ## Populate it with random spins
     <a href='vscode-notebook-cell:/Users/caballero/repos/teaching/phy415fall23/content/4_distributions/activity-ising_model.ipynb#W2sZmlsZQ%3D%3D?line=20'>21</a> for row in range(cellLength):


NameError: name 'np' is not defined
\end{verbatim}

\begin{Shaded}
\begin{Highlighting}[]
\CommentTok{\#\# Hold onto the values of the magnetization }
\CommentTok{\#\# for each step in the simulation}
\NormalTok{magnetizationArray }\OperatorTok{=}\NormalTok{ np.zeros(simulationSteps)}

\CommentTok{\#\# Monte Carlo Loop}
\ControlFlowTok{for}\NormalTok{ step }\KeywordTok{in} \BuiltInTok{range}\NormalTok{(simulationSteps):}
    
    \CommentTok{\#\# Store the magnetization at this step}
\NormalTok{    magnetizationArray[step] }\OperatorTok{=}\NormalTok{ magnetizationAtStep}
    
    \CommentTok{\#\# Store the energy before swapping the spin randomly}
\NormalTok{    oldEnergy }\OperatorTok{=}\NormalTok{ energyAtStep}
    
    \CommentTok{\#\# Select a spin from the cell}
\NormalTok{    ithSpin }\OperatorTok{=}\NormalTok{ random.randrange(cellLength)}
\NormalTok{    jthSpin }\OperatorTok{=}\NormalTok{ random.randrange(cellLength)}
    
    \CommentTok{\#\# Flip the spin of that one site}
\NormalTok{    spinArray[ithSpin,jthSpin] }\OperatorTok{=} \OperatorTok{{-}}\NormalTok{spinArray[ithSpin,jthSpin]}
    
    \CommentTok{\#\# Calculate the energy after that change}
\NormalTok{    energyAtStep }\OperatorTok{=}\NormalTok{ calculateEnergy(spinArray)}
\NormalTok{    deltaE }\OperatorTok{=}\NormalTok{ energyAtStep }\OperatorTok{{-}}\NormalTok{ oldEnergy}
    
    \CommentTok{\#\# If the change resulted in an increase in the total energy,}
    \CommentTok{\#\# evaluate whether to accept the value or not}
    \ControlFlowTok{if}\NormalTok{ deltaE }\OperatorTok{\textgreater{}} \FloatTok{0.0}\NormalTok{:}
        
\NormalTok{        probabilityOfFlip }\OperatorTok{=}\NormalTok{ np.exp(}\OperatorTok{{-}}\NormalTok{deltaE}\OperatorTok{/}\NormalTok{temperature)}
        
        \CommentTok{\#\# If the the random value is lower than the probability,}
        \CommentTok{\#\# reverse the change to the spin, and recalculate the energy}
        \ControlFlowTok{if}\NormalTok{ random.random()}\OperatorTok{\textgreater{}}\NormalTok{probabilityOfFlip:}
            
\NormalTok{            spinArray[ithSpin,jthSpin] }\OperatorTok{=} \OperatorTok{{-}}\NormalTok{spinArray[ithSpin,jthSpin]}
\NormalTok{            energyAtStep }\OperatorTok{=}\NormalTok{ oldEnergy}
            \ControlFlowTok{continue}
        
\NormalTok{    magnetizationAtStep }\OperatorTok{=}\NormalTok{ np.}\BuiltInTok{sum}\NormalTok{(spinArray)}
\end{Highlighting}
\end{Shaded}

\begin{verbatim}
---------------------------------------------------------------------------

NameError                                 Traceback (most recent call last)

/Users/caballero/repos/teaching/phy415fall23/content/4_distributions/activity-ising_model.ipynb Cell 4 line 3
      <a href='vscode-notebook-cell:/Users/caballero/repos/teaching/phy415fall23/content/4_distributions/activity-ising_model.ipynb#W3sZmlsZQ%3D%3D?line=0'>1</a> ## Hold onto the values of the magnetization 
      <a href='vscode-notebook-cell:/Users/caballero/repos/teaching/phy415fall23/content/4_distributions/activity-ising_model.ipynb#W3sZmlsZQ%3D%3D?line=1'>2</a> ## for each step in the simulation
----> <a href='vscode-notebook-cell:/Users/caballero/repos/teaching/phy415fall23/content/4_distributions/activity-ising_model.ipynb#W3sZmlsZQ%3D%3D?line=2'>3</a> magnetizationArray = np.zeros(simulationSteps)
      <a href='vscode-notebook-cell:/Users/caballero/repos/teaching/phy415fall23/content/4_distributions/activity-ising_model.ipynb#W3sZmlsZQ%3D%3D?line=4'>5</a> ## Monte Carlo Loop
      <a href='vscode-notebook-cell:/Users/caballero/repos/teaching/phy415fall23/content/4_distributions/activity-ising_model.ipynb#W3sZmlsZQ%3D%3D?line=5'>6</a> for step in range(simulationSteps):
      <a href='vscode-notebook-cell:/Users/caballero/repos/teaching/phy415fall23/content/4_distributions/activity-ising_model.ipynb#W3sZmlsZQ%3D%3D?line=6'>7</a>     
      <a href='vscode-notebook-cell:/Users/caballero/repos/teaching/phy415fall23/content/4_distributions/activity-ising_model.ipynb#W3sZmlsZQ%3D%3D?line=7'>8</a>     ## Store the magnetization at this step


NameError: name 'np' is not defined
\end{verbatim}

\begin{Shaded}
\begin{Highlighting}[]
\NormalTok{plt.figure(figsize}\OperatorTok{=}\NormalTok{(}\DecValTok{8}\NormalTok{,}\DecValTok{8}\NormalTok{))}\OperatorTok{;}
\NormalTok{c }\OperatorTok{=}\NormalTok{ plt.pcolor(spinArray, cmap}\OperatorTok{=}\StringTok{\textquotesingle{}Greys\textquotesingle{}}\NormalTok{)}\OperatorTok{;}
\NormalTok{plt.axis(}\StringTok{\textquotesingle{}square\textquotesingle{}}\NormalTok{)}\OperatorTok{;}
\end{Highlighting}
\end{Shaded}

\begin{verbatim}
---------------------------------------------------------------------------

NameError                                 Traceback (most recent call last)

/Users/caballero/repos/teaching/phy415fall23/content/4_distributions/activity-ising_model.ipynb Cell 5 line 1
----> <a href='vscode-notebook-cell:/Users/caballero/repos/teaching/phy415fall23/content/4_distributions/activity-ising_model.ipynb#W4sZmlsZQ%3D%3D?line=0'>1</a> plt.figure(figsize=(8,8));
      <a href='vscode-notebook-cell:/Users/caballero/repos/teaching/phy415fall23/content/4_distributions/activity-ising_model.ipynb#W4sZmlsZQ%3D%3D?line=1'>2</a> c = plt.pcolor(spinArray, cmap='Greys');
      <a href='vscode-notebook-cell:/Users/caballero/repos/teaching/phy415fall23/content/4_distributions/activity-ising_model.ipynb#W4sZmlsZQ%3D%3D?line=2'>3</a> plt.axis('square');


NameError: name 'plt' is not defined
\end{verbatim}

\begin{Shaded}
\begin{Highlighting}[]

\end{Highlighting}
\end{Shaded}

plt.figure(figsize=(8,6))

plt.plot(magnetizationArray) plt.ylabel(`Magnetization')
plt.xlabel(`Simulation Steps')

\begin{Shaded}
\begin{Highlighting}[]

\end{Highlighting}
\end{Shaded}
