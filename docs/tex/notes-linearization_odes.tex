\section{19 Sep 23 - Notes: Heading towards
Chaos}\label{sep-23---notes-heading-towards-chaos}

One of the principal motivations for our work thus far has been to get
to analyze systems that show suprising and complex behavior. In
mechanics, you might have heard of those kinds of behviors as chaotic or
exhibiting chaos. It turns out the chaos is a specific kind of behavior
that we can find. To motivate us, let's remind ourselves of how cool
nature can be.

\href{https://inv.tux.pizza/watch?v=fDek6cYijxI}{\pandocbounded{\includegraphics[keepaspectratio]{https://markdown-videos-api.jorgenkh.no/youtube/fDek6cYijxI?width=720&height=405}}}

\begin{itemize}
\tightlist
\item
  Non-Commercial Link: \url{https://inv.tux.pizza/watch?v=fDek6cYijxI}
\item
  Commercial Link: \url{https://youtube.com/watch?v=fDek6cYijxI}
\end{itemize}

\subsection{Reminders of the Large Angle
Pendulum}\label{reminders-of-the-large-angle-pendulum}

Last time we investigated the phase portrait of the large angle
pendulum, we we could arrive at by re-writing the differential equation

\[
\ddot{\theta} = -\dfrac{g}{L}\sin(\theta)
\]

as 2 first-order differential equations:

\[
\dot{\theta} = \omega \hspace{0.5in}\text{and}\hspace{0.5in} \dot{\omega} = -\frac{g}{L}\sin(\theta)
\]

By setting both of these equations equal to zero simultaneously, we also
argued that this system has
(\href{https://faculty.math.illinois.edu/~kapovich/417-16/card.pdf}{countably})
infinite fixed points at \((n\pi, 0)\) for \(n\in \mathbb{Z}\) in
\((\theta,\omega)\) phase space.

\subsubsection{Linearization and the
Jacobian}\label{linearization-and-the-jacobian}

Now we turn to the challenge of characterizing these fixed points with
the linearization of the system (see the end of tuesday's activity for
some more notes on this). Recall that we can do this by finding the
eigenvalues of the Jacobian Matrix of the system at its fixed point. For
the system \(\dot{x} = f(x,y)\), \(\dot{y} = g(x,y)\) the jacobian
matrix looks like this:

\[
A = \begin{bmatrix} \frac{\partial f}{\partial x} & \frac{\partial f}{\partial y} \\ \frac{\partial g}{\partial x} & \frac{\partial g}{\partial y}\end{bmatrix}_{(x^*,y^*)}
\]

\textbf{✅ Try this}

Calculate the general Jacobian matrix \(A\) for this system, then
calculate what it is at the fixed point \((0,0)\).

We have the Jacobian at \((0,0)\) now but we still need to find its
eigenvalues. Let's remind ourselves about eigenvalues and eigenvectors.

\subsection{Eigenvalues}\label{eigenvalues}

Eigenvalues and the closely related Eigenvectors are indispensible in
physics, math, and computational science. These ideas for the basis (pun
somewhat intened) for countless problems, from the
\href{https://phys.libretexts.org/Bookshelves/Nuclear_and_Particle_Physics/Introduction_to_Applied_Nuclear_Physics_(Cappellaro)/02\%3A_Introduction_to_Quantum_Mechanics/2.04\%3A_Energy_Eigenvalue_Problem}{energy
eigenvalue equation} that is the foundation of quantum mechanics, to the
stability of complex nonlinear systems, to Normal Modes of oscillators.
\emph{Eigenproblems} show up all over in physics.

A brief tangent: Once some scientists were using an eigenvalue driven
algorithm called principal component analysis to study the genes of
people that live in Europe. They found that these eigenvalues/vectors
reproduced a map of Europe with surprising accuracy
(\href{https://www.ncbi.nlm.nih.gov/pmc/articles/PMC2735096/}{link}). So
these tools are extremely, and often unreasonably powerful.

Eigenvalues are the \(\lambda\) in the equation:

\[
A\mathbf{v} = \lambda \mathbf{v}
\]

Where \(A\) is a linear operator of the vector space that \(\mathbf{v}\)
lives in. In finite-dimensional vector spaces, rhese linear operators
are always matrices. There is a bit of physical intuition behind this
equation: An eigenvector of \(A\) is a vector that only gets stretched
or squished by \(\lambda\) when \(A\) acts on \(\mathbf{v}\).

\subsubsection{Video from 3Blue1Brown}\label{video-from-3blue1brown}

Grant Sanderson makes a lot of great videos on mathematics and
statistics. This one about eigenvalues and eigenvectors is particularly
good.

\href{https://inv.tux.pizza/watch?v=PFDu9oVAE-g}{\pandocbounded{\includegraphics[keepaspectratio]{https://markdown-videos-api.jorgenkh.no/youtube/PFDu9oVAE-g?width=720&height=405}}}

\begin{itemize}
\tightlist
\item
  Non-Commercial Link: \url{https://inv.tux.pizza/watch?v=PFDu9oVAE-g}
\item
  Commercial Link: \url{https://youtube.com/watch?v=PFDu9oVAE-g}
\end{itemize}

\subsubsection{Finding Eigenvalues}\label{finding-eigenvalues}

To actually find the eigenvalues of a matrix, you solve the
\textbf{characteristic polynomial} of the matrix, which you obtain by
solving the equation:

\[
|A - \lambda I | = 0 
\]

Where the vertical bars means determinant.

To find Eigenvectors, simply plug in the values you found for
\(\lambda\) into the original eigenvalue equation
\(A\mathbf{v} = \lambda \mathbf{v}\), using
\(\mathbf{v} = \begin{bmatrix}x \\ y\end{bmatrix}\). You'll find some
simple relationship between \(x\) and \(y\). Any scalar multiple of an
eigenvector is also an eigenvector so we usually just choose the
simplest one. Say if you found that \(x = -y\). Then for a nice clean
looking eigenvector you could choose
\(\begin{bmatrix} -1 \\ 1\end{bmatrix}\).

\textbf{✅ Try this}

Analytically, find the eigenvalues of the Jacobian matrix you calculated
earlier. Use the below bullets to identify these eigenvalues with the
type of the fixed point.

\begin{itemize}
\tightlist
\item
  \$\mathrm{Re}(\lambda) \textgreater{} 0 \$ for both eigenvalues:
  Repeller/Source (unstable)
\item
  \$\mathrm{Re}(\lambda) \textless{} 0 \$ for both eigenvalues:
  Attractor/Sink (stable)
\item
  One eigenvalue positive, one negative: Saddle
\item
  Both eigenvalues pure imaginary: Center
\end{itemize}

Note: You can actually learn quite a bit more from this analysis, see
Strogatz chapter 6. In fact, these eigenvalues can tell us about the
nature of the local trajectories.

\subsection{Additional Resources}\label{additional-resources}

\subsubsection{Stability and Eigenvalues of a linearized
system}\label{stability-and-eigenvalues-of-a-linearized-system}

Steve Brunton at UW does dynamical systems research and has a great
video that demonstrates how we make meaning of those eigenvalues.

\href{https://inv.tux.pizza/watch?v=XXjoh8L1HkE}{\pandocbounded{\includegraphics[keepaspectratio]{https://markdown-videos-api.jorgenkh.no/youtube/XXjoh8L1HkE?width=720&height=405}}}

\begin{itemize}
\tightlist
\item
  Non-Commercial Link: \url{https://inv.tux.pizza/watch?v=XXjoh8L1HkE}
\item
  Commercial Link: \url{https://youtube.com/watch?v=XXjoh8L1HkE}
\end{itemize}

\subsubsection{Eigenvalues,
Computationally}\label{eigenvalues-computationally}

We can use \texttt{np.linalg.eig()} to find the eigenvalues (and
normalized eigenvectors) of a matrix which we represent as numpy array.
Below is some doe that does this (note the imaginary unit is represented
as \(j\) in python):

\begin{Shaded}
\begin{Highlighting}[]
\ImportTok{import}\NormalTok{ numpy }\ImportTok{as}\NormalTok{ np}
\NormalTok{A }\OperatorTok{=}\NormalTok{ np.array([[}\DecValTok{0}\NormalTok{,}\DecValTok{1}\NormalTok{],[}\OperatorTok{{-}}\DecValTok{1}\NormalTok{,}\DecValTok{0}\NormalTok{]])}
\NormalTok{eigvals }\OperatorTok{=}\NormalTok{ np.linalg.eig(A)[}\DecValTok{0}\NormalTok{]}
\NormalTok{eigvecs }\OperatorTok{=}\NormalTok{ np.linalg.eig(A)[}\DecValTok{1}\NormalTok{]}

\BuiltInTok{print}\NormalTok{(}\StringTok{"eigenvalues:"}\NormalTok{, eigvals)}
\end{Highlighting}
\end{Shaded}

\begin{verbatim}
eigenvalues: [0.+1.j 0.-1.j]
\end{verbatim}

This can be super handy when you just need to do some quick
characterization from the eigenvalues of a matrix. However, be warned -
since you only get numerical answers you can lose quite a bit of the
nuance that comes from if you had calculated these. We'll see how that
can be an issue later in the semester when we tackle normal modes.

\subsubsection{Handwritten Notes}\label{handwritten-notes}

\begin{itemize}
\tightlist
\item
  \href{../../assets/notes/Notes-Linearization_ODEs.pdf}{Linearization
  of ODEs}
\item
  \href{../../assets/notes/Notes-Linearization_Example.pdf}{Linearization
  example}
\end{itemize}
