\section{14 \& 16 Nov 23 - Using the FFT with Real
Data}\label{nov-23---using-the-fft-with-real-data}

Y'all have learned the main process for using an FFT on a 1D signal, so
let's use it in the context of some real world data. We'll give you
different kind of data that you can choose to work with: Transit light
curve data and audio signals. Take a look at both datasets and start
working on whichever one you'd like first.

\subsection{Transit Data}\label{transit-data}

You work in a lab that is observing light curves to determine if there
are transiting objects near a star and, if so, what is their period.
Using the brightness of the star itself, we can determine it's mass and
then it's a quick analysis to get the mass of the transiting objects
(Thanks, Newton!).

In this activity, you have three data files that describe the CCD
voltage on a sensor as a function of time. The observations were taken
over a 48 hour period. You will need to read in the data, determine the
sampling frequency, and then develop an FFT model to look into the
frequency components.

\subsubsection{Data Files}\label{data-files}

\begin{itemize}
\tightlist
\item
  \href{https://raw.githubusercontent.com/dannycab/phy415msu/main/MMIPbook/assets/data/FFT/obs1.csv}{Observation
  1}
\item
  \href{https://raw.githubusercontent.com/dannycab/phy415msu/main/MMIPbook/assets/data/FFT/obs2.csv}{Observation
  2}
\item
  \href{https://raw.githubusercontent.com/dannycab/phy415msu/main/MMIPbook/assets/data/FFT/obs3.csv}{Observation
  3}
\end{itemize}

The first data file is known to contain data from the observation of 2
transiting objects. Your lab mate misnamed the other two files, so it's
not clear if they are the same set of observations or not.

Start with observation 1. You can use \texttt{pd.read} from the pandas
library to read in the data.

\textbf{✅ Questions to answer}

For observation 1,

\begin{enumerate}
\def\labelenumi{\arabic{enumi}.}
\tightlist
\item
  What does the FFT look like? Can you describe where the real
  observavtions might be in the plot?
\item
  Can you clean the noise from the data to find the real signal?
\item
  Can you estimate the transit times for the objects?
\end{enumerate}

For observations 2 and 3,

\begin{enumerate}
\def\labelenumi{\arabic{enumi}.}
\tightlist
\item
  Which one (or both or neither) observations are those of observation
  1?
\item
  If there's a file with new observations, can you learn the same things
  as above?
\end{enumerate}

\begin{Shaded}
\begin{Highlighting}[]
\CommentTok{\#\# your code here}
\end{Highlighting}
\end{Shaded}

\subsection{Audio Signals \& Effects}\label{audio-signals-effects}

A lot of audio processing boils down to taking the FFT of some audio
signal, doing something to that frequency spectrum, the using the
inverse FFT to get a new signal back with some effects on it. Here's
some musical data to practice this:

Data:

\begin{itemize}
\tightlist
\item
  Signal 1:
  \href{https://raw.githubusercontent.com/valentine-alia/phy415fall23/main/content/assets/note.wav}{a
  single note played on a guitar}
\item
  Signal 2:
  \href{https://raw.githubusercontent.com/valentine-alia/phy415fall23/main/content/assets/chord.wav}{a
  chord played on a guitar}
\item
  Signal 3:
  \href{https://raw.githubusercontent.com/valentine-alia/phy415fall23/main/content/assets/riff.wav}{metal
  riff without distortion}
\item
  Signal 4:
  \href{https://raw.githubusercontent.com/valentine-alia/phy415fall23/main/content/assets/track.wav}{a
  full track}
\end{itemize}

For some code to help you read-in and listen to the audio, refer to
\href{https://dannycaballero.info/phy415fall23/content/3_waves/notes-Using-FFTs.html}{yesterday's
notes}

\textbf{✅ Questions to answer}

For signal 1:

\begin{enumerate}
\def\labelenumi{\arabic{enumi}.}
\tightlist
\item
  What does the fourier transform of the signal look like? Does it have
  a lot of peaks?
\item
  What note is being played?
\end{enumerate}

For signal 2:

\begin{enumerate}
\def\labelenumi{\arabic{enumi}.}
\tightlist
\item
  What chord is being played (i.e.~what are the individual notes?)
\item
  After taking the FFT of the data, try to get rid of the frequency
  components of the highest note in the chord then take the IFFT to see
  what the new signal sounds like. What do you notice?
\item
  Repeat 2. but for the lowest note.
\end{enumerate}

For signal 3:

\begin{enumerate}
\def\labelenumi{\arabic{enumi}.}
\tightlist
\item
  Without distortion, this metal riff sounds\ldots{} kindof lame to be
  honest. Try adding distortion to the signal.
\end{enumerate}

For signal 4:

\begin{enumerate}
\def\labelenumi{\arabic{enumi}.}
\tightlist
\item
  Oftentimes in audio enginnering its useful to use an ``equalizer'' to
  boost or lessen certain frequencies in an audio signal to give it a
  different sound. Try cutting out all the frequencies above some
  threshold and see what it sounds like then.
\item
  Try to isolate the synth bass from the rest of the signal. Is it
  possible to get a perfect separation?
\item
  At the begging of the track, you can hear that there is some effect on
  the guitars. Can you use the FFT to figure out what this effect is
  doing?
\end{enumerate}

\begin{Shaded}
\begin{Highlighting}[]
\CommentTok{\#\# your code here}
\end{Highlighting}
\end{Shaded}
