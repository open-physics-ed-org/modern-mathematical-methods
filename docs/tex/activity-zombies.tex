\section{7 Dec 23 - Activity:
Modeling}\label{dec-23---activity-modeling}

While this course has focused on physics and physical systems, the same
techniques and approaches that you have developed, can be used to
investigate systems in other fields. In this activity, we will
demonstrate an initial model, and then ask you to complicate it with the
ideas you have developed. Consider the critical issue of a potential
\textbf{zombie apocalypse}.

While this might seem a little silly, the models and dynamics of a
zombie apocalypse can be adapted from the models of disease spread. In
fact, the
\href{https://www.usatoday.com/story/news/nation/2021/03/05/zombie-apocalypse-cdc-useful-advice-any-emergency-pandemic/6920614002/}{CDC
had a page dedicated} to this topic. For those who might be really
interested, the behavior fo zombie hoards can be modeled like flocking
birds or schools of fish.

\subsection{What are we doing?}\label{what-are-we-doing}

Saving the world, obviously.

\href{https://inv.tux.pizza/watch?v=MQ8ZKw7YIfQ}{\pandocbounded{\includegraphics[keepaspectratio,alt={Night of the Living Dead}]{https://markdown-videos-api.jorgenkh.no/youtube/MQ8ZKw7YIfQ?width=720&height=405}}}

\begin{itemize}
\tightlist
\item
  Non-Commercial Link: \url{https://inv.tux.pizza/watch?v=MQ8ZKw7YIfQ}
\item
  Commercial Link: \url{https://youtube.com/watch?v=MQ8ZKw7YIfQ}
\end{itemize}

\subsection{Where do we start? Compartmental
models}\label{where-do-we-start-compartmental-models}

The modeling of epidemics follows from a few simple considerations:

\begin{itemize}
\tightlist
\item
  the population under consideration ca be sufficiently
  compartmentalized into distinct groups, and
\item
  there is a rate (fixed or variable) at which individuals can move
  between compartments.
\end{itemize}

The
\href{https://en.wikipedia.org/wiki/Compartmental_models_in_epidemiology}{compartmental
model} is a simple way to model the spread of a disease and where we
will start.

\subsubsection{The SIS model}\label{the-sis-model}

The Susceptible-Infected-Susceptible (SIS) model describes an epidemic
in which healthy individuals (Susceptibles) can become infected with the
disease in question (Infected). The infected individuals can be cured
but retain no natural immunity to the disease (i.e., become susceptible
again). The figure below illustrates the compartments and flux of
individuals in the SIS model. Such models are suitable for bacterial
infections.

\begin{figure}
\centering
\pandocbounded{\includegraphics[keepaspectratio,alt={The SIS model}]{../images/activity-zombies_sis.png}}
\caption{The SIS model}
\end{figure}

The dynamics are given by,

\[\dfrac{dS}{dt} = -\beta S I + \gamma I\]
\[\dfrac{dI}{dt} = \beta S I - \gamma I\]

The rate at which susceptibles are infected is related to the contact
rate (\(\beta\)). The recovery rate (\(\gamma\)) describes how quickly
individuals are cured. The rate equations that govern this model are two
ordinary non-linear differential equations. Here we have neglected the
birth and death rates of the populations by making the approximation the
epidemic occurs very quickly. If we assume there's a total number of
people \(N\), then we can write these equations as:

\[\dfrac{dS}{dt} = -\beta S I + \gamma (N - S)\] \[I(t) = N - S(t)\]

For given values of \(\beta\) and \(\gamma\), we can solve these
equations numerically.

\subsubsection{The SIR model}\label{the-sir-model}

The Susceptible-Infected-Recovered (SIR) model describes an epidemic in
which healthy individuals (Susceptibles) can become infected with the
disease in question (Infected). The infected individuals can be cured
and retain a natural immunity to the disease (Recovered). The figure
below illustrates the compartments and flux of individuals in the SIR
model. Such models are suitable for viral infections. We can modify this
model for the zombies.

\begin{figure}
\centering
\pandocbounded{\includegraphics[keepaspectratio,alt={The SIR model}]{../images/activity-zombies_sir.png}}
\caption{The SIR model}
\end{figure}

The transfer rates can be defined for the SIR model as was for the SIS
model. The rate equations that govern this model are three ordinary
non-linear differential equations. Again, we have neglected the birth
and death rates of the populations by making the approximation the
epidemic occurs very quickly.

\[\dfrac{dS}{dt} = -\beta S I\]
\[\dfrac{dI}{dt} = \beta S I - \gamma I\] \[\dfrac{dR}{dt} = \gamma I\]

Again, for given values of \(\beta\) and \(\gamma\), we can solve these
equations numerically.

\subsubsection{The SZR model}\label{the-szr-model}

This SZR model treats the population as compartmentalized into the 3
groups: Susceptible (S), Zombie (Z), and Removed (R). Movement between
these groups is illustrated below.

\begin{figure}
\centering
\pandocbounded{\includegraphics[keepaspectratio,alt={The SZR model}]{../images/activity-zombies_szr.png}}
\caption{The SZR model}
\end{figure}

Susceptibles can become zombies through an encounter with a zombie.
Zombies can move to the removed compartment by being destroyed in
classic manners (e.g., Dawn of the Dead). Susceptibles can move to the
removed compartment through death by a non-zombie encounter. Finally,
removed individuals can become zombies through typical resurrection
techniques (e.g., Live and Let Die).

The deterministic model includes transfer rates we have seen before, but
we include the rates of transfer from susceptible to removed
(\(\delta\)) and removed to zombie (\(\xi\)).

\[\dfrac{dS}{dt} = - \beta S Z - \delta S\]
\[\dfrac{dZ}{dt} = \beta S Z + \xi R - \gamma S Z\]
\[\dfrac{dR}{dt} = \delta S +\gamma SZ - \xi R\]

For given values of values of \(\beta\), \(\gamma\), \(\delta\), and
\(\xi\), we can solve these equations numerically.

\subsubsection{Tasks for today}\label{tasks-for-today}

\textbf{✅ Do this}

\begin{enumerate}
\def\labelenumi{\arabic{enumi}.}
\tightlist
\item
  Work with your group to model the SIS and SIR systems. Choose values
  or non-dimensionalize the equations to reduce the number of
  parameters.
\item
  Consider the many tools we have used to investigate ODEs. What more
  information can you extract from these models? What are the
  limitations of these models?
\item
  Do the same for the SZR model. What are the limitations of this model?
  What are the advantages of this model?
\end{enumerate}

\begin{Shaded}
\begin{Highlighting}[]
\CommentTok{\#\# your code here}
\end{Highlighting}
\end{Shaded}

\subsubsection{Challenges}\label{challenges}

\textbf{✅ Do this}

\begin{enumerate}
\def\labelenumi{\arabic{enumi}.}
\setcounter{enumi}{3}
\tightlist
\item
  \textbf{Challenge}: Build an SZQR model where the CDC quarantines some
  number of susecptibles and zombies (at different rates). How does this
  change the dynamics of the system?
\item
  \textbf{Challenge}: Introduce stochasticity into the model. What
  happened if instead there were a probability of infection or death?
  How does this change the dynamics of the system? If you are interested
  in this, look at \href{../assets/papers/z-epidemic.pdf}{this paper}.
\end{enumerate}

\begin{Shaded}
\begin{Highlighting}[]
\CommentTok{\#\# your code here}
\end{Highlighting}
\end{Shaded}

\begin{Shaded}
\begin{Highlighting}[]

\end{Highlighting}
\end{Shaded}
