\section{17 Oct 23 - Activity: Normal
Modes}\label{oct-23---activity-normal-modes}

As we begin to work with multi-particle systems, we begin to open the
number of degrees of freedom of the system. We increase the complexity
of it and the challenge of modeling it. We will increasingly need tools
that help us take less tractable issues and make them palatable. One
such tool that forms the basis of many others is \textbf{Normal Modes}.

Normal modes are ways of breaking up our understanding of a system into
discrete (potentially infinite and hopefully vanishingly small) pieces.
This lets us get at the big picture of the system and characterize how
much we are neglecting. The concept of normal modes underlies much of
the analysis done in quantum mechanics, field theory, fluid mechanics,
and signal processing. In fact, numerical routines often make use of
forms of these modes to do economize calculations -- \emph{especially
when we move beyond a few hundred bodies in the system}. This stuff is
part and parcel of most research in physics. For example, how we
understand the
\href{https://en.wikipedia.org/wiki/Cosmic_background_radiation}{Cosmic
Background Radiation} relies on a
\href{https://www.quantamagazine.org/how-ancient-light-reveals-the-universes-contents-20200128/}{normal
mode like analysis}.

\subsection{Canonical Coupled
Oscillators}\label{canonical-coupled-oscillators}

There is still plenty of room for pencil and paper. We will explore this
numerically later.

We will start with a form of guided lecture and build the understanding
of these methods together. Then you will apply what you learned to a new
system.

Let's assume you have a chain of two mass connected by springs (all the
same \(k\)) as below.

\begin{figure}
\centering
\pandocbounded{\includegraphics[keepaspectratio,alt={Coupled Oscillator set up. Two oscillators connected by three springs in a horizontal line.}]{../images/activity-CoupledOsc-NormalModes_two-coupled-gliders-diagram.png}}
\caption{Coupled Oscillator set up. Two oscillators connected by three
springs in a horizontal line.}
\end{figure}

\subsubsection{Deriving the Equations of
Motion}\label{deriving-the-equations-of-motion}

Let's write the Lagrangian for this system to get equations of motion.
The kinetic and potential energies are:

\[T=\frac{1}{2}m\dot{x}_1^2+\frac{1}{2}m\dot{x}_2^2\]
\[U=\frac{1}{2}k{x_1}^2+\frac{1}{2}k(x_2-x_1)^2 + \frac{1}{2}k{x_2}^2\]

Thus the Lagrangian is for this system is:

\[L = \frac{1}{2}m\dot{x}_1^2+\frac{1}{2}m\dot{x}_2^2 - \frac{1}{2}k{x_1}^2-\frac{1}{2}k(x_2-x_1)^2 - \frac{1}{2}k{x_2}^2\]

Which produces two equations of motion:

\[\frac{d}{dt}\left(\frac{\partial L}{\partial \dot{x}_1}\right) - \frac{\partial L}{\partial x_1} = 0\]
\[\frac{d}{dt}\left(\frac{\partial L}{\partial \dot{x}_2}\right) - \frac{\partial L}{\partial x_2} = 0\]

Which we can write as:

\[m\ddot{x}_1 = -kx_1 + k(x_2-x_1)\]
\[m\ddot{x}_2 = -kx_2 + k(x_1-x_2)\]

Or

\[\ddot{x}_1 = -\frac{2k}{m}x_1 + \frac{k}{m}x_2\]
\[\ddot{x}_2 = -\frac{2k}{m}x_2 + \frac{k}{m}x_1\]

\subsubsection{Finding Normal Modes}\label{finding-normal-modes}

\paragraph{Method 1: Assume a
Solution}\label{method-1-assume-a-solution}

A normal mode solution is one in which all parts of a system oscillate
with the same frequency. For linear systems, this can be a very useful
approach because any solution can be represented as a linear combination
of normal modes. In general,

\[f(t) = \sum_n c_n \psi_n(t)\]

where \(f(t)\) is the solution, \(c_n\) are constants, and \(\psi_n(t)\)
are the normal modes. This kind of normal mode analysis forms the basis
for signal analysis, image processing, and many other fields. For
non-linear systems, we lose superposition and thus, we lose out ability
to employ normal modes. However, near equilibria and in other limits,
normal modes are still used. We will start by assuming a normal mode
solution for both masses and plug that into our equations of motion.

Assume:

\[x_1(t) = A_1\cos(\omega t + \phi_1)\]
\[x_2(t) = A_2\cos(\omega t + \phi_2)\]

where \(A_1\), \(A_2\), \(\phi_1\), and \(\phi_2\) are constants to be
determined. We can plug these into our differential equations and find
the normal modes. We made a chose for the form of the solution. Other
forms are acceptable as long as they have two free parameters; it's a
second order differential equation.

We can plug these into our differential equations and get:

\[-\omega^2A_1\cos(\omega t + \phi_1) = -\frac{k}{m}A_1\cos(\omega t + \phi_1) + \frac{k}{m}(A_2\cos(\omega t + \phi_2)-A_1\cos(\omega t + \phi_1))\]
\[-\omega^2A_2\cos(\omega t + \phi_2) = -\frac{k}{m}A_2\cos(\omega t + \phi_2) + \frac{k}{m}(A_1\cos(\omega t + \phi_1)-A_2\cos(\omega t + \phi_2))\]

which we can collect the terms \(A_1\) and \(A_2\) and get:

\[-\omega^2A_1\cos(\omega t + \phi_1) + 2\frac{k}{m}A_1\cos(\omega t + \phi_1) = \frac{k}{m}A_2\cos(\omega t + \phi_2)\]
\[-\omega^2A_2\cos(\omega t + \phi_2) + 2\frac{k}{m}A_2\cos(\omega t + \phi_2) = \frac{k}{m}A_1\cos(\omega t + \phi_1)\]

Which we rewrite as:

\[\left(2\frac{k}{m}-\omega^2\right)A_1\cos(\omega t + \phi_1) = \frac{k}{m}A_2\cos(\omega t + \phi_2)\]
\[\frac{k}{m}A_1\cos(\omega t + \phi_1)=\left(2\frac{k}{m}-\omega^2\right)A_2\cos(\omega t + \phi_2)\]

Assuming non-zero solutions, we can divide the equations:

\[\dfrac{\left(2\frac{k}{m}-\omega^2\right)}{\frac{k}{m}} = \dfrac{\frac{k}{m}}{\left(2\frac{k}{m}-\omega^2\right)}\]

So that,

\[\left(2\frac{k}{m}-\omega^2\right)^2 = \left(\frac{k}{m}\right)^2\]

Or,

\[\omega^2 = \frac{2k}{m} \pm \frac{k}{m}\]

So, if solution is a normal mode, there's actually two of them it could
be: one with \(\omega = \sqrt{\frac{3k}{m}}\) and one with
\(\omega = \sqrt{\frac{k}{m}}\).

\textbf{✅ Do this}

\begin{enumerate}
\def\labelenumi{\arabic{enumi}.}
\tightlist
\item
  Given the frequencies of the normal mode solutions, find the relative
  amplitudes of the two masses for each mode. How do the masses move
  relative to each other?
\item
  Consider the following initial conditions, the left mass (\(x_1\)) is
  displaced by \(x_0\). Everything else is at rest. Find the particular
  solution for each mass (\(x_1\) and \(x_2\)).
\item
  Plot their motion.
\item
  (later, if time) develop a numerical solution for \(x_1\) and \(x_2\)
  for any choice of conditions.
\end{enumerate}

\begin{Shaded}
\begin{Highlighting}[]
\CommentTok{\#\# Your code here}
\end{Highlighting}
\end{Shaded}

\paragraph{Method 2: Matrix Analysis (the Eigenvalue
Problem)}\label{method-2-matrix-analysis-the-eigenvalue-problem}

That analysis to find the normal modes was a bit of a pain in algebra.
As we add more objects, that pain will grow. We can use a matrix
approach to make this easier. It's also a systematic approach that as we
will see plays nicely with computing.

Starting from the equations of motion:

\[\ddot{x}_1 = -\frac{2k}{m}x_1 + \frac{k}{m}x_2\]
\[\ddot{x}_2 = -\frac{2k}{m}x_2 + \frac{k}{m}x_1\]

We rewrite them as a set of matrix equations for the vector
\(\mathbf{x} = \begin{bmatrix}x_1\\x_2\end{bmatrix}\):

\[\ddot{\mathbf{x}} = \begin{bmatrix}-\frac{2k}{m} & \frac{k}{m}\\\frac{k}{m} & -\frac{2k}{m}\end{bmatrix}\mathbf{x}\]

This is a form of the equation
\(\ddot{\mathbf{x}} = \pmb{A} \mathbf{x}\) where \(\pmb{A}\) is the
coefficient matrix. We can solve this equation by finding the
eigenvalues and eigenvectors of \(\pmb{A}\). The eigenvalues are the
frequencies of the normal modes and the eigenvectors are the relative
amplitudes of the masses. The reason is that the determinant of the
coefficient matrix when there is a normal mode solution vanishes. That
is,

\[\det(\pmb{A}+\omega^2\pmb{I}) = 0\]

where \(\pmb{I}\) is the identity matrix. This is a polynomial equation
in \(\omega^2\) and the roots of that polynomial are the normal mode
frequencies. This technique is very common in theoretical physics with
different guesses for the solution forms giving rise to different
eigenvalue problems (e.g.,
\href{https://en.wikipedia.org/wiki/Bessel_function}{Bessel functions
for 2D surface problems} or
\href{https://en.wikipedia.org/wiki/Hermite_polynomials}{Hermite
polynomials for the Quantum Harmonic Oscillator}.

\textbf{✅ Do this}

\begin{enumerate}
\def\labelenumi{\arabic{enumi}.}
\tightlist
\item
  Using the ``determinant of the coefficient matrix'' (really the
  eigenvalues of \(\pmb{A}\) solved by
  \(\det(\pmb{A}+\omega^2\pmb{I})\)), find the normal mode \(\omega^2\).
\item
  Find the eigenvectors of the coefficient matrix. These are the
  relative amplitudes of the masses.
\item
  These results should agree with the previous method. Do they?
\item
  Plot the normal modes solutions.
\end{enumerate}

\begin{Shaded}
\begin{Highlighting}[]
\CommentTok{\#\# your code here}
\end{Highlighting}
\end{Shaded}

\subsection{Applying Normal Modes to a New
System}\label{applying-normal-modes-to-a-new-system}

\subsubsection{Two vertical pendulums connected by a
spring}\label{two-vertical-pendulums-connected-by-a-spring}

Consider two vertical pendulums of length \(l\) connected via their
masses \(M\) by a weak spring \(k\). By weak, we mean that the spring
constant is small. See below for a canonical setup:

\begin{figure}
\centering
\pandocbounded{\includegraphics[keepaspectratio,alt={Coupled pendulua diagram}]{../images/activity-CoupledOsc-NormalModes_coupled_diag.jpg}}
\caption{Coupled pendulua diagram}
\end{figure}

\textbf{✅ Do this}

\begin{enumerate}
\def\labelenumi{\arabic{enumi}.}
\tightlist
\item
  In this limit, write down the pair of second order linear differential
  equations for the horizontal motion of each pendulum bob around its
  equilibrium.
\item
  Find and describe the normal modes. Use plots of your choosing to
  explain what you found.
\end{enumerate}

\begin{Shaded}
\begin{Highlighting}[]
\CommentTok{\#\# your code here}
\end{Highlighting}
\end{Shaded}

\subsection{Three Coupled Oscillators}\label{three-coupled-oscillators}

Consider the setup below consisting of three masses connected by springs
to each other. We intend to find the normal modes of the system by
denoting each mass's displacement (\(x_1\), \(x_2\), and \(x_3\)).

\begin{figure}
\centering
\pandocbounded{\includegraphics[keepaspectratio,alt={3 Coupled Oscillators}]{../images/activity-CoupledOsc-NormalModes_3_coupled_osc.png}}
\caption{3 Coupled Oscillators}
\end{figure}

\subsection{Finding the Normal Mode
Frequencies}\label{finding-the-normal-mode-frequencies}

\textbf{✅ Do this}

This is not magic as we will see, it follows from our choices of
solution. Here's the steps and what you might notice about them:

\begin{enumerate}
\def\labelenumi{\arabic{enumi}.}
\tightlist
\item
  Guess what the normal modes might look like? Write your guesses down;
  how should the masses move? (It's ok if you are not sure about all of
  them, try to determine one of them)
\item
  Write down the energy for the whole system, \(T\) and \(U\) (We have
  done this before, but not for this many particles)
\item
  Use the Euler-Lagrange Equation to find the equations of motion for
  \(x_1\), \(x_2\), and \(x_3\). (We have done this lots, so make sure
  it feels solid)
\item
  Reformulate the equations of motion as a matrix equation
  \(\ddot{\mathbf{x}} = \mathbf{A} \mathbf{x}\). What is \(\mathbf{A}\)?
  (We have done this, but only quickly, so take your time)
\item
  Consider solutions of the form \(Ce^{i{\omega}t}\), plug that into
  \(x_1\), \(x_2\), and \(x_3\) to show you get
  \(\mathbf{A}\mathbf{x} = -\omega^2 \mathbf{x}\). (We have not done
  this, we just assumed it works! It's ok if this is annoying, we only
  have to show it once.)
\item
  Find the normal mode frequencies by taking the determinant of
  \(\mathbf{A} - \mathbf{I}\lambda\). Note that this produces the
  following definition: \(\lambda = -\omega^2\) (We have not done this
  together and we can if it's confusing.)
\end{enumerate}

\subsection{Finding the Normal Modes
Amplitudes}\label{finding-the-normal-modes-amplitudes}

Ok, now we need to find the normal mode amplitudes. That is we assumed
sinusoidal oscillations, but at what amplitudes? We will show how to do
this with one frequency (\(\omega_1\)), and then break up the work of
the the other two. These frequencies are:

\[\omega_A = 2\dfrac{k}{m}; \qquad \omega_B = \left(2-\sqrt{2}\right)\dfrac{k}{m}; \qquad \omega_C = \left(2+\sqrt{2}\right)\dfrac{k}{m}\qquad\]

\textbf{✅ Do this}

After we do the first one, pick another frequencies and repeat. Answer
the follow questions:

\begin{enumerate}
\def\labelenumi{\arabic{enumi}.}
\tightlist
\item
  What does this motion physically look like? What are the masses doing?
\item
  How does the frequency of oscillation make sense? Why is it higher or
  lower than \(\omega_A\)?
\end{enumerate}

\begin{itemize}
\tightlist
\item
  \href{../assets/notes/Notes-Three_Coupled_Oscillators.pdf}{Partial
  Solution to Activity}
\end{itemize}
